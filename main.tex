\documentclass[a4paper]{article}

\usepackage{hyperref}

\newcommand{\triposcourse}{Complex Analysis}
\newcommand{\triposterm}{Lent 2020}
\newcommand{\triposlecturer}{Prof. N. Wickramasekera}
\newcommand{\tripospart}{IB}

\usepackage{amsmath}
\usepackage{amssymb}
\usepackage{amsthm}
\usepackage{mathrsfs}

\usepackage{tikz-cd}

\theoremstyle{plain}
\newtheorem{theorem}{Theorem}[section]
\newtheorem{lemma}[theorem]{Lemma}
\newtheorem{proposition}[theorem]{Proposition}
\newtheorem{corollary}[theorem]{Corollary}
\newtheorem{problem}[theorem]{Problem}
\newtheorem*{claim}{Claim}

\theoremstyle{definition}
\newtheorem{definition}{Definition}[section]
\newtheorem{conjecture}{Conjecture}[section]
\newtheorem{example}{Example}[section]

\theoremstyle{remark}
\newtheorem*{remark}{Remark}
\newtheorem*{note}{Note}

\title{\triposcourse{}
\thanks{Based on the lectures under the same name taught by \triposlecturer{} in \triposterm{}.}}
\author{Zhiyuan Bai}
\date{Compiled on \today}

%\setcounter{section}{-1}

\begin{document}
    \maketitle
    This document serves as a set of revision materials for the Cambridge Mathematical Tripos Part \tripospart{} course \textit{\triposcourse{}} in \triposterm{}.
    However, despite its primary focus, readers should note that it is NOT a verbatim recall of the lectures, since the author might have made further amendments in the content.
    Therefore, there should always be provisions for errors and typos while this material is being used.
    \tableofcontents
    \section{Basic Notions}
\subsection{Definitions}
\begin{definition}
    A subset $U\subset\mathbb C$ is open if $\forall u\in U,\exists r>0,D_r(u)\subset U$.
\end{definition}
If we identify $\mathbb C$ with $\mathbb R^2$ in the obivous way, and give $\mathbb R^2$ the usual topology, then $U\subset\mathbb C$ is open iff it is open in $\mathbb R^2$.\\
We shall be interested in functions $f:U\to\mathbb C$ where $U$ is open.
\begin{definition}
    Let $U\subset\mathbb C$ be open and $f:U\to\mathbb C$.
    We say $\lim_{z\to c}f(z)=A$ if $\forall\epsilon>0,\exists\delta>0$,
    $$0<|z-c|<\delta\implies |f(z)-f(c)|<\epsilon$$
    $f$ is continuous at $c\in U$ if $\lim_{z\to c}f(z)=f(c)$.
    $f$ is continuous in $U$ if it is continuous everywhere in $U$.
\end{definition}
We can always write $f(x+iy)=u(x,y)+iv(x,y)$ where $u,v:\mathbb U\to\mathbb R$ where $U$ is open in $\mathbb R^2$.
It is easy to see that $\mathbb C$ inherits the continuity condition in $\mathbb R^2$ since it inherits the topology from it.
So we have that the continuity of $f$ is equivalent to that of $u,v$.
\begin{definition}
    Let $f:U\to\mathbb C$ be as before, and $w\in U$.
    We say $f$ is differentiable at $w$ if the limit
    $$\lim_{z\to w}\frac{f(z)-f(w)}{z-w}$$
    exists.
    If it exists, we say $f$ is differentiable at $w$ and has derivative equals the limit.\\
    If $f$ is differentiable everywhere in an open neighbourhood of $w$, then we say $f$ is holomorphic at $w$.
    \footnote{Some authors use the word `analytic'}\\
    We say $f$ is holomorphic on $U$ if it is holomorphic everywhere on $U$.
    (Or equivalently it is differentiable everywhere.)
\end{definition}
Rules that can be obtained from real differentiation by first principle mostly extends to complex differentiation.
For example, polynomials are differentiable everywhere, and rational functions are differentiable in the subset (which is open as the complement of a finite subset) where they are defined.
\subsection{The Cauchy-Riemann Equation}
Although complex differentiation exhibits similar definition as real differentiation, they behave very differently.
A natural question is, is the differentiability of $f$ behave the same as the differentiability of $u,v$?
The answer is no.
\begin{theorem}[Cauchy-Riemann Equation]
    $f:U\to\mathbb C$ is differentiable at $w=c+id$ iff the functions $u,v$ are differentiable at $(c,d)$ and they satisfies $u_x=v_y,u_y=-v_x$.
\end{theorem}
\begin{proof}
    By definition, $f$ is differentiable at $w$ with derivative $f^\prime(w)=p+iq$ if and only if
    $$\lim_{z\to w}\frac{f(z)-f(w)-f^\prime(z)(z-w)}{|z-w|}=0$$
    which is equivalent to the case where we seperate the real and imaginary part, which is just to say
    $$
    \begin{cases}
        \lim_{(x,y)\to(c,d)}\frac{u(x,y)-u(c,d)-(p(x-c)-q(y-d))}{\|(x-c,y-d)\|}=0\\
        \lim_{(x,y)\to(c,d)}\frac{v(x,y)-v(c,d)-(p(x-c)+q(y-d))}{\|(x-c,y-d)\|}=0
    \end{cases}$$
    which happens iff $Du(c,d)=(p,-q)$ and $Dv(c,d)=(p,q)$.
    The theorem follows.
\end{proof}
\begin{remark}
    Just because $u,v$ has partial derivatives satisfying Cauchy-Riemann equation does not guarantee the total differentiability.
    A deeper question is what if we require them to be so on an open set, but that goes beyond the scope of the course.
\end{remark}
From the theorem, we have obtained an expression of the complex derivative $f^\prime=u_x+iv_x$.
If we just want to show that the differentiability of $f$ at $c+id$ implies the existence of partials of $u,v$ satisfying Cauchy-Riemann equations, then we can just proceed by taking the limit of $z-w\to 0$ from both axes.
\begin{example}
    $f(z)=\bar{z}$ is nowhere differentiable since it does not satisfy the first C-R equation.
\end{example}
\begin{remark}
    Complex differentiability is much more restrictive than real ones.
    Of course, we will exhibit examples to justify it.
\end{remark}
\begin{theorem}[Liouville Theorem]\label{holo_bdd_const}
    If $f:\mathbb C\to \mathbb C$ is holomorphic and bounded, then $f$ is constant.
\end{theorem}
\begin{theorem}\label{infinite_holo}
    Let $U$ be an open set in $\mathbb C$, then if $f:U\to\mathbb C$ is holomorphic, so is $f^\prime$.
\end{theorem}
\begin{theorem}\label{uniform_holo}
    A sequence of holomorphic functions $f_n:U\to\mathbb C$ on the same open domain, and $f_n\to f$ uniformly, then $f$ is holomorphic.
\end{theorem}
By Theorem \ref{infinite_holo}, holomorphic functions are infinite differentiable, so any order of partial derivatives of $u,v$ exists, so we can differentiate the Cauchy-Riemann equation one more time to get $u_{xx}+u_{yy}=0$, so $u$ is a harmonic function, similarly $v$ is harmonic as well.
So the real and imaginary part of a holomorphic function are harmonic.
\begin{corollary}
    Let $f=u+iv:U\to\mathbb C$, suppose that $u,v$ has continuous partial derivatives in $U$ and $u,v$ satisfy Cauchy-Riemann Equation, then $f$ is holomorphic.
\end{corollary}
\begin{proof}
    Immediate from what we have got and a result from Analysis and Topology.
\end{proof}
\begin{remark}
    Once we know Theorem \ref{infinite_holo}, the converse of the preceding corollary follows.
\end{remark}
\begin{definition}
    A curve is a continuous map $\gamma:[a,b]\to\mathbb C$.
    It is called $C^1$ if $\gamma^\prime$ exists and is continuous on $[a,b]$.\\
    An open subset $U\subset\mathbb C$ is path connected if for any $z,w\in U$, there is a curve $\gamma:[a,b]\to U$ such that $\gamma(a)=z,\gamma(b)=w$.\\
    A non-empty open path-connected subset of $\mathbb C$ is called a domain.
\end{definition}
\begin{corollary}
    Let $U$ be a domain and $f:U\to\mathbb C$ is homomorphic. If $f^\prime\equiv 0$, then $f$ is constant.
\end{corollary}
\begin{proof}
    Proved in Analysis and Topology.
\end{proof}
\subsection{Power Series}
\begin{theorem}
    For any sequence of complex numbers $(c_n)$, there is some $R\in [0,\infty]$ such that the power series
    $$\sum_{n=0}^\infty c_n(z-a)^n$$
    such that it converges absolutely for $|z-a|<R$ and diverges for $|z-a|>R$.\\
    If $R\in\mathbb R_{>0}$, and if $0<r<R$, then the convergence is uniform on $B_r(a)$ (or $D_r(a)$, they are equivalent anyways).
\end{theorem}
\begin{definition}
    Such an $R$ is called the radius of convergence.
\end{definition}
\begin{proposition}
    $R=\frac{1}{\lambda}$ where $\lambda=\limsup_{n\to\infty}\sqrt[n]{|c_n|}$.
\end{proposition}
\begin{theorem}
    Define $f$ on the disk $D_R(a)$ by
    $$f(z)=\sum_{n=0}^\infty c_n(z-a)^n$$
    where $R>0$ is the radius of convergence of the series.
    Then\\
    1. $f$ is holomorphic on this disk,\\
    2. And
    $$\sum_{n=0}^\infty (n+1)c_{n+1}(z-a)^n$$
    is convergent on $D_R(a)$ and is the derivative of $f$.\\
    3. $f$ has derivatives of all orders on $D_R(a)$ and $f^{(n)}(a)=c_nn!$.\\
    4. If $f$ vanished in some open disk $D_r(a)$ with $0<r<R$, then it vanishes on the whole of $D_R(a)$.
\end{theorem}
\begin{proof}
    3 follows from 2 and 4 follows from 3.\\
    WLOG $a=0$.
    To prove 1 and 2, let $|z|<R$ and choose $\rho$ such that $|x|<\rho<R$, then we have
    $$\lim_{n\to\infty}\frac{n|c_n||z|^{n-1}}{|c_n|\rho^n}=\lim_{n\to\infty}\frac{n}{\rho}\left( \frac{|z|}{\rho} \right)^n\to 0$$
    so the derived series also converges on $D_R(0)$.\\
    To show that $f$ is holomorphic with derivative equals the derived series, we fix some $w\in D_R(0)$.
    Note that $f$ is differentiable at $w$ with derivative being the derived series $\sigma$ if and only if the function
    $$g(h)=
    \begin{cases}
        \frac{f(z)-f(w)}{z-w}\text{, if $z\neq w$}
        \sigma\text{, if $z=w$}
    \end{cases}$$
    is continuous at $w$.
    But $g(z)=\sum_{n=0}^\infty h_n(z)$ where
    $$h_n(z)=\begin{cases}
        c_n(z^{n-1}+z^{n-2}w+\cdots+w^{n-1})\text{, for $z\neq w$}\\
        nc_nw^{n-1}\text{, for $z=w$}
    \end{cases},n>0;h_0(z)=0$$
    Note that each $h_n$ is continuous.
    So the continuity of $g$ at $w$ follows from the uniform convergence of the series.
    Again we take $0<|w|<r<R$, so for any point $z\in D_r(0)$, we have
    $$|h_n(z)|=|c_n||z^{n-1}+z^{n-2}w+\cdots+w^{n-1}|\le n|c_n||r|^{n-1}$$
    which proves the uniform convergence.
    So 1 and 2 are proved.
\end{proof}
\begin{definition}
    The exponential function is defined as
    $$e^z=\exp(z)=\sum_{k=0}^\infty\frac{z^k}{k!}$$
\end{definition}
\begin{definition}
    An entire function is a function that is holomorphic on all of $\mathbb C$.
\end{definition}
\begin{proposition}
    1. $\exp$ is entire with derivative equals to itself.\\
    2. $\exp(z+w)=\exp(z)\exp(w)$.\\
    3. $\exp(z)\neq 0$.\\
    4. $\exp(z)=1\iff z\in2\pi i\mathbb Z$.\\
    5. $\exp:\mathbb C\to\mathbb C\setminus\{0\}$ is surjective.
\end{proposition}
\begin{proof}
    Trivial.
\end{proof}
\begin{definition}
    Given $z\in\mathbb C$, we say $w\in\mathbb C$ is a logarithm of $z$ of $e^w=z$.
\end{definition}
Note that $z$ has a logarithm iff $z\neq 0$.
If $w_1,w_2$ are two logrithms of $z$, then we immediately have $w_1-w_2\in 2\pi i\mathbb Z$.
\begin{definition}
    Let $U\subset\mathbb C^\star=\mathbb C\setminus\{0\}$.
    A branch of the logarithm on $U$ is a continuous function $\lambda:U\to\mathbb C$ such that $\exp\circ\lambda=\operatorname{id}$.
\end{definition}
\begin{remark}
    1. If such $\lambda$ does exixt, then $\lambda$ is holomorphic on $U$ by simply computation.\\
    2. From the definition, it follows that $|z|=e^{\operatorname{Re}\lambda(z)}$, so it is immediate that any branch $\lambda$ has $\operatorname{Re}\lambda(z)=\log|z|$.
\end{remark}
\begin{definition}
    The principal branch of logarithm is the function $\operatorname{Log}:U_1=\mathbb C\setminus\{x\in\mathbb R:x\le 0\}\to\mathbb C$ given by $\operatorname{Log}(z)=\log|z|+i\arg z$ where the value of $\arg$ is taken in the open interval $(-\pi,\pi)$. 
\end{definition}
Obviously in this open set it is a inverse of $\exp$.
To see the continuity, observe that the map $z\mapsto z/|z|$ is continuous and the map $\theta\mapsto e^{i\theta}$ is a homeomorphism $(-\pi,\pi)\to S\setminus\{-1\}$
\begin{proposition}
    1. $\operatorname{Log}$ is holomorphic on $U_1$ with derivative $1/z$.\\
    2. For $|z|<1$,
    $$\operatorname{Log}(z)=\sum_{n=1}^\infty\frac{(-1)^{n+1}z^n}{n}$$
\end{proposition}
\begin{proof}
    Trivial.
\end{proof}
\begin{remark}
    There is no way to extend $\operatorname{Log}$ to the whole of $\mathbb C^\star$ and remain a branch since
    $$\lim_{\theta\to \pi^+}e^{i\theta}\neq\lim_{\theta\to \pi^-}e^{i\theta}$$
    In fact there is no branch that can be such extended.
\end{remark}
Using $\exp$ and $\operatorname{Log}$, we can construct the familiar functions we had in real analysis, like the trigonometrics, hypertrigonometrics, etc., in the way we know (like $\cos(z)=(e^{iz}+e^{-iz})/2$, etc).
We can also define, for $z\notin \mathbb R_{\le 0}$, $z^\alpha=\exp(\alpha\operatorname{Log}(z))$.\\
Let $f:U\to\mathbb C$ be a holomorphic function.
A number $w\in U$ having $f^\prime(w)$ is nice in the sense that $f$ is invertible there by Inverse Function Theorem with $(f^{-1})^\prime(f(w))=1/f^\prime(w)$.
One can check that $f^{-1}$ also satisfies the Cauchy-Riemann equation, hence is also holomorphic.\\
It is also nice in a geometric sense as $f$ preserves angles at that point.
Let $\gamma_1,\gamma_2:[-1,1]\to\mathbb C$ be $C^1$ curves with $\gamma_1(0)=\gamma_2(0)=w$ and $\gamma_1^\prime(0)\gamma_2^\prime(0)\neq 0$, then since $(f\circ \gamma_1)^\prime(0)=f^\prime(w)\gamma^\prime_1(0)$.
Similar for $\gamma_2$.
Since $f^\prime(w)\neq 0$, we have
$$\frac{\gamma^\prime_1(0)}{\gamma_2^\prime(0)}=\frac{(f\circ\gamma_1)^\prime(0)}{(f\circ\gamma_2)^\prime(0)}$$
So $f$ preserves angles.
\begin{definition}[Conformal Equivalence]
    Let $U$ be an open set.
    A holomorphic function $f:U\to\mathbb C$ is called conformal at $w\in U$ if $f^\prime(w)\neq 0$.\\
    If for domains $D,D'$ there is a holomorpic bijection $f:D\to\tilde{D}$ which is conformal at every point, then we say $D,D'$ are conformal equivalent.
\end{definition}
\begin{example}
    1. Any Mobius map $f:\mathbb C\cup\{\infty\}\to\mathbb C\cup\{\infty\}$ is a conformal equivalence.\\
    2. The map $z\mapsto z^n$ from $\{\mathbb z\in\mathbb C^\star:0\le\arg z\le \pi/n\}$ to the upper half plane $\mathbb H=\{z\in\mathbb C:\operatorname{Im}z>0\}$ is conformal.\\
    3. The exponential $\exp:\{z\in\mathbb C:-\pi<\operatorname{Im}(z)<\pi\}\to\mathbb C\setminus\mathbb R_{\le 0}$.\\
    4. Consider $g:z\mapsto (z-i)/(z+i)$ with $g:\mathbb H\to D_1(0)$.
    It is also conformal.
\end{example}
\begin{theorem}[Riemann Mapping]
    Let $D$ be a domain bounded by a simple closed curve, then it is conformally equivalent to the unit disk.\\
    More generally, any simply connected domain that is not the whole complex plane is conformally equivalent to the unit disk.
\end{theorem}
    \section{Complex Integration}
\subsection{Definition}
We want to generalize the notion of the real Riemann integration to the integration to complex valued functions on the complex plane.
\begin{definition}
    If $f:[a,b]\to\mathbb C$ is continuous, then we define the integral of $f$ to be
    $$\int_a^b f(x)\,\mathrm dx :=\int_a^b\operatorname{Re}f(x)\,\mathrm dx+i\int_a^b\operatorname{Im}f(x)\,\mathrm dx$$
\end{definition}
Easy to check that both integrals are well-defined and the integral is linear.
\begin{proposition}
    $$\left|\int_a^bf(t)\,\mathrm dt\right|\le (b-a)\sup_{t\in [a,b]}|f(t)|$$
\end{proposition}
\begin{proof}
    If the integral is zero then there is nothing to prove.
    Otherwise we can write it in the form $re^{i\theta}$ for some $r\in\mathbb R_{>0},\theta\in\mathbb R$, so
    \begin{align*}
        \left|\int_a^bf(t)\,\mathrm dt\right|=r&=\int_a^be^{-i\theta}f(t)\,\mathrm dt\\
        &=\int_a^b\operatorname{Re}(e^{-i\theta}f(t))\,\mathrm dt\\
        &\le \int_a^b|\operatorname{Re}(e^{-i\theta}f(t))|\,\mathrm dt\\
        &\le \int_a^b|f(t)|\,\mathrm dt\\
        &\le (b-a)\sup_{t\in [a,b]}|f(t)|
    \end{align*}
    As desired.
\end{proof}
Note that the equality holds iff $f$ is constant.
\begin{definition}
    Let $\gamma:[a,b]\to\mathbb C$ be a $C^1$ curve, then the length of $\gamma$ is
    $$\int_a^b|\gamma^\prime(t)|\,\mathrm dt$$
    Also this curve is called simple iff $\gamma(t_1)=\gamma(t_2)\iff t_1\equiv t_2\pmod{b-a}$
\end{definition}
\begin{definition}
    Let $\gamma:[a,b]\to\mathbb C$ be a $C^1$ curve and $f:U\to\mathbb C$ be continuous, then we define the integral of $f$ over $\gamma$ by
    $$\int_\gamma f(t)\,\mathrm dt=\int_a^b(f\circ\gamma)(t)\gamma^\prime(t)\,\mathrm dt$$
\end{definition}
One can check that
\begin{proposition}
    1.
    $$\left(\int_{\gamma}f\right)+\alpha\left(\int_\gamma g\right)=\int_\gamma(f+\alpha g)$$
    2. 
    $$\int_{\gamma\pm\delta}=\int_\gamma\pm\int_\delta$$
    3. Let $\gamma,\delta$ be two parameterizations of the same curve linked by an injective $C^1$ function, then
    $$\int_\gamma=\int_\delta$$
\end{proposition}
Where the addition and substraction of paths are defined the way a sensible person would expect.
\begin{proof}
    Trivial.
\end{proof}
\begin{definition}
    For continuous piecewise $C^1$ curves $\gamma=\gamma_1+\gamma_2+\cdots+\gamma_n$, we set
    $$\int_\gamma=\sum_{k=1}^n\int_{\gamma_k}$$
\end{definition}
Note that by additivity of the integral over paths, this is well-defined.
\begin{proposition}
    For any continuous function $f:U\to\mathbb c$ and any (piecewise $C^1$) curve $\gamma:[a,b]\to U$, we have
    $$\left|\int_\gamma f(z)\,\mathrm dz\right|\le\sup_{t\in[a,b]}|f(\gamma(t))|\int_a^b|\gamma^\prime(t)|\,\mathrm dt$$
\end{proposition}
\begin{proof}
    Suffices to show the case when $\gamma$ is $C^1$, then
    \begin{align*}
        \left|\int_\gamma f(z)\,\mathrm dz\right|&=\left|\int_a^bf(\gamma(t))\gamma^\prime(t)\,\mathrm dt\right|\\
        &\le\int_a^b|f(\gamma(t))||\gamma^\prime(t)|\,\mathrm dt\\
        &\le\sup_{t\in[a,b]}|f(\gamma(t))|\int_a^b|\gamma^\prime(t)|\,\mathrm dt
    \end{align*}
    As desired.
\end{proof}
\subsection{Cauchy's Theorem}
\begin{theorem}[Fundamental Theorem of Calculus for Complex Integrals]
    For a continuous function $f:U\to\mathbb C$, if there is a holomorphic $F:U\to\mathbb C$ such that $F^\prime=f$, then for any (piecewise $C^1$) curve $\gamma:[a,b]\to\mathbb C$ we have
    $$\int_\gamma f(z)\,\mathrm dz=F(\gamma(b))-F(\gamma(a))$$
\end{theorem}
In particular we have
$$\oint_\gamma f(z)\,\mathrm dz=0$$
For closed $\gamma$.
\begin{proof}
    Again suffices to consider $\gamma$ as $C^1$, then we have
    $$\int_\gamma f(z)\,\mathrm dz=\int_a^bF^\prime(\gamma(t))\gamma^\prime(t)\,\mathrm dt=\int_a^b(F\circ\gamma)^\prime(t)\,\mathrm dt=F(\gamma(b))-F(\gamma(a))$$
    Done.
\end{proof}
\begin{example}
    For $\gamma:[0,\pi]\to\mathbb C$ by $t\mapsto Re^{2it}$, we have for $n\neq -1$
    $$\oint_\gamma z^n\,\mathrm dz=0$$
    Indeed it is the derivative of the holomorphic function $z^{n+1}/(n+1)$.\\
    But for $n=-1$,
    $$\oint_\gamma z^{-1}\,\mathrm dz=2\pi R$$
    which is nonzero, hence it does not have an antiderivative defined on any open set containing $\gamma$, so logarithm has no branch on $\mathbb C^\star$.
\end{example}
What is interesting is the converse of the theorem.
\begin{theorem}
    Let $U$ be a path-connected open set, and $f:U\to\mathbb C$ be continuous.
    If for any (piecewise $C^1$) closed curve $\gamma:[a,b]\to U$ we have
    $$\oint_\gamma f(z)\,\mathrm dz$$
    then $f$ has an antiderivative on $U$.
\end{theorem}
\begin{proof}
    Fix $\alpha\in U$.
    Consider the function $F:U\to\mathbb C$ with
    $$F(z)=\int_\gamma f(z)\,\mathrm dz$$
    where $\gamma$ is a piecewise $C^1$ curve on $U$ connecting $\alpha$ and $z$.
    To see the existence of $\gamma$, we know by path-connectedness of $U$ that there is a continuous curve from $\alpha$ to $z$, then we can construct a piecewise $C^1$ one by a compactness argument.\\
    Then such an $F$ is well defined by our condition.
    One can simply check to see that $F$ is holomorphic and $F^\prime=f$.
\end{proof}
\begin{definition}
    A domain $U$ is star-shaped if $\exists s\in U$ such that any other $x\in U$, there is a straight line joining $x$ and $s$.
\end{definition}
\begin{definition}
    A triangle $T$ is the convex hull of three non-colinear points on the complex plane, so
    $$T(z_1,z_2,z_3)=\{az_1+bz_2+cz_3:a,b,c\in [0,1],a+b+c=1\}$$
    We denote by $\partial T$ the boundary of $T$, which is the union of three line segments, and we choose it to be with anticlockwise direction. 
\end{definition}
\begin{corollary}
    In any star-shaped domain, if for any triangle $T$ in the domain we have
    $$\oint_{\partial T}f(z)\,\mathrm dz=0$$
    then $f$ admits a holomorphic antiderivative.
\end{corollary}
\begin{proof}
    Basically the same proof but take $\alpha=s$ and take the path to be a straight line.
\end{proof}
\begin{theorem}[Cauchy's Theorem]
    For a holomorphic $f$ and closed $\gamma$,
    $$\oint_\gamma f(z)\,\mathrm dz=0$$
\end{theorem}
\begin{theorem}[Cauchy's Theorem for Triangles]
    Let $U\subset C$ be open and $f:U\to\mathbb C$ holomorphic.
    If $T$ is a triangle in $U$, then
    $$\oint_{\partial T}f(z)\,\mathrm dz=0$$
\end{theorem}
\begin{proof}
    We subdivide the triangles by joining the midpoints, so we disassembles $T$ into $4$ smaller triangles.
    Call them $T^1,T^2,T^3,T^4$, and the directions of their boundaries are consistently given (anticlockwise), so we have
    $$\oint_{\partial T}f(z)\,\mathrm dz=\sum_{k=1}^4\oint_{\partial T^k}f(z)\,\mathrm dz$$
    We set
    $$\eta(T)=\oint_{\partial T}f(z)\,\mathrm dz$$
    So $\eta(T)=\sum_{k}\eta(T^k)$, hence there is some $k$ such that $|\eta(T^k)|\ge|\eta(T)|/4$, and $\operatorname{length}(\partial T^k)=\operatorname{length}(\partial T)/2$, so we repeat this process to get a nested sequence of triangles $T=T_0\supset T_1\supset T_2\supset\cdots$ such that $|\eta(T_k)|/4\le|\eta(T_{k+1})|$ and $\operatorname{length}(\partial T_{k+1})=\operatorname{length}(\partial T_k)/2$.
    But each $T_k$ is closed and the diameter goes to $0$, hence $\bigcap_kT_k=\{z_0\}$ for some $z_0\in\mathbb C$.
    For any $\epsilon>0$, there is some $\delta>0$ such that $|z-z_0|<\delta\implies |f(z)-f(z_0)-f^\prime(z_0)(z-z_0)|<\epsilon|z-z_0|$
    For $n$ large enough, we have $T_n\subset D_\delta(z_0)$.
    \begin{align*}
        |\eta(T_n)|&=\left|\oint_{\partial T_n}f(z)-(f(z_0)+f^\prime(z_0)(z-z_0))\,\mathrm dz\right|\\
        &\le\sup_{z\in \delta T_n}|f(z)-(f(z)+f^\prime(z_0)(z-z_0)|\operatorname{length}(\partial T_n)\\
        &\le\epsilon\sup_{z\in\partial T_n}|z-z_0|\operatorname{length}(\partial T_n)\\
        &\le\epsilon(\operatorname{length}(\partial T_n))^2
    \end{align*}
    So
    $$\frac{\eta(T)}{4^n}\le\epsilon(\operatorname{length}(\partial T^n))^2=\frac{\epsilon(\operatorname{length}(\partial T))^2}{4^n}\implies\forall\epsilon>0,\eta(T)\le \epsilon(\operatorname{length}(\partial T))^2$$
    Hence we must have $\eta(T)=0$.
\end{proof}
\begin{theorem}
    Let $f:U\to\mathbb C$ be continuous.
    If $S\subset U$ is a finite set and if $f$ is holomorphic in $U\setminus S$, then
    $$\oint_{\partial T}f(z)\,\mathrm dz=0$$
    for any triangle $T\subset U$.
\end{theorem}
\begin{proof}
    Subdivide $T$ into $N=4^n$ parts as before to $T_1,T_2,\ldots,T_N$, then $|I|\le 6|S|$ where $I=\{j:T_j\cap S\neq\varnothing\}$.
    Hence we have, by Cauchy's Theorem on triangles,
    \begin{align*}
        \left|\oint_{\partial T}f(z)\,\mathrm dz\right|&=\left|\sum_{j\in I}\oint_{\partial T_j}f(z)\,\mathrm dz\right|\\
        &\le\sum_{j\in I}\sup_{z\in\partial T_j}|f(z)|\operatorname{length}(\partial T_j)\\
        &\le 6|S|\sup_{z\in\partial T}|f(z)|\operatorname{length}(\partial T)\frac{1}{2^n}
    \end{align*}
    Letting $n\to\infty$ finishes the proof.
\end{proof}
\begin{corollary}[Cauchy's Theorem on Star-Shaped Domains]
    Let $U\subset C$ be a star-shaped domain and $f:U\to\mathbb C$ be continuous, and holomorphic on $U\setminus S$ where $S$ is a finite set.
    Then
    $$\oint_\gamma f(z)\,\mathrm dz=0$$
    for any closed curve on $U$.
\end{corollary}
\begin{proof}
    Follows directly.
\end{proof}
\subsection{Cauchy Integral Formula and Consequences}
\begin{theorem}[Cauchy's Integral Formula for a disk]
    Let $D=D_r(a)$, and let $f:D\to\mathbb C$ be holomorphic, then for any $0<\rho<r$ and any $w\in D_\rho(a)$, we have
    $$f(w)=\frac{1}{2\pi i}\oint_{\partial D_\rho(a)}\frac{f(z)}{z-w}\,\mathrm dz$$
    where $\partial D_\rho(a)$ denotes the curve $[0,1]\ni t\mapsto a+\rho e^{2\pi it}$
\end{theorem}
In particular,
$$f(a)=\int_0^1 f(a+\rho e^{2\pi it})\,\mathrm dt$$
This is known as the mean-value property.
\begin{proof}
    We have
    $$\oint_{\partial D_\rho(a)}\frac{f(z)-f(w)}{z-w}\,\mathrm dz=0$$
    by the preceding theorem.
    So we have
    \begin{align*}
        \oint_{\partial D_\rho(a)}\frac{f(z)}{z-w}\,\mathrm dz&=f(w)\oint_{\partial D_\rho(a)}\frac{1}{z-w}\,\mathrm dz\\
        &=f(w)\oint_{\partial D_\rho(a)}\frac{1}{z-a}\,\mathrm dz+f(w)\oint_{\partial D_\rho(a)}\sum_{n=1}^\infty\frac{(w-a)^n}{(z-a)^{n+1}}\,\mathrm dz\\
        &=f(w)2\pi i+f(w)\sum_{n=1}^\infty(w-a)^n\oint_{\partial D_\rho(a)}\frac{1}{(z-a)^{n+1}}\,\mathrm dz\\
        &=f(w)2\pi i
    \end{align*}
    Note that we can change the order of integration and summation since the series (as a geometric series) converges uniformly (which is easy enough to prove).
\end{proof}
So we can prove that a bounded entire function is constant.
\begin{proof}[Proof of Theorem \ref{holo_bdd_const}]
    Let $f$ be a bounded entire function.
    It suffice to assume that $f$ has sublinear growth since it is bounded.
    So $|f(z)|\le C(1+|z|^\alpha)$ for some $C\ge 0$ and $\alpha\in (0,1)$.
    Let $w\in\mathbb C$, by Cauchy integral formula, for any $\rho>|w|$ we have
    $$f(w)=\frac{1}{2\pi i}\oint_{D_\rho(0)}\frac{f(z)}{z-w}\,\mathrm dz$$
    Also
    $$f(0)=\frac{1}{2\pi i}\oint_{D_\rho(0)}\frac{f(z)}{z}\,\mathrm dz$$
    Hence
    \begin{align*}
        |f(w)-f(0)|&=\left|\frac{1}{2\pi i}\oint_{D_\rho(0)}f(z)\left(\frac{1}{z-w}-\frac{1}{z}\right)\,\mathrm dz\right|\\
        &=\frac{|w|}{2\pi}\left|\oint_{D_\rho(0)}\frac{f(z)}{z(z-w)}\,\mathrm dz\right|\\
        &\le|w|\rho\sup_{z\in D_\rho(0)}\frac{|f(z)|}{|z||z-w|}\\
        &=\sup_{z\in D_\rho(0)}\frac{|w||f(z)|}{|z-w|}\\
        &\le\sup_{z\in D_\rho(0)}\frac{|w|C(1+\rho^\alpha)}{|z-w|}\to 0
    \end{align*}
    As $\rho\to\infty$.
    Hence $f(w)=f(0)$.
\end{proof}
\begin{corollary}
    Every nonconstant polynomial with complex coefficient has a root in $\mathbb C$.
\end{corollary}
\begin{proof}
    Assume there is a complex polynomial $p(z)=a_nz^n+\cdots+a_0$ with $n\ge1,a_n\neq 0$ which has no root, then $p$ is never $0$, hence $1/p$ is entire.
    So it suffices to show that $p$ is bounded.
    Note that for $z\neq 0$ we have $|p(z)|=|z^n||a_n+a_{n-1}z^{-1}+\cdots+a_0z^{-n}|\to\infty$ as $|z|\to\infty$.
    Hence we can choose $R>0$ such that $p(z)\ge 1$ for any $|z|>R$, so $p$ is bounded outside the disk $B_R(0)$, but it is also bounded inside it since $B_R(0)$ is compact.
    So by Theorem \ref{holo_bdd_const}, $p$ must be a constant, contradiction.
\end{proof}
\begin{theorem}[Local Maximum Modulus Principle]
    Let $f:D_r(a)\to\mathbb C$ be holomorphic.
    If $|f(z)|\le|f(a)|$ for any $z\in D_r(a)$, then $f$ is constant.
\end{theorem}
\begin{proof}
    We shall use the mean-value property.
    $$f(a)=\int_0^1 f(a+\rho e^{2\pi it})\,\mathrm dt$$
    for any $0<\rho<r$.
    So
    $$|f(a)|=\left|\int_0^1 f(a+\rho e^{2\pi it})\,\mathrm dt\right|\le\sup_{t\in [0,1]}|f(a+\rho e^{2\pi it})|\le|f(a)|$$
    So automatically all inequality must be equality, therefore $|f(a+\rho e^{2\pi it})|$ must be constant.
    But this works for all $\rho\in (0,r)$, so $|f|$ is constant, which implies $f$ being constant by Cauchy-Riemann equation.
\end{proof}
\begin{theorem}[Taylor Series]
    Let $f:D_r(a)\to\mathbb C$ be holomorphic.
    Then $f$ has a convergent power series representation on $D_a(r)$ in the form
    $$f(w)=\sum_{n=0}^\infty c_n(w-a)^n,c_n=\frac{1}{2\pi i}\oint_{\partial D_\rho(a)}\frac{f(z)}{(z-a)^{n+1}}\,\mathrm dz$$
    for $|w|<\rho<r$.
\end{theorem}
Once we have established this, we immediately know that $f$ is infinitely differentiable on $D_a(r)$, therefore $c_n=f^{(n)}(a)/n!$.
\begin{proof}
    For any $w\in D_r(a)$ and any $\rho$ with $|w-a|<\rho<r$ we have
    \begin{align*}
        f(w)&=\frac{1}{2\pi i}\oint_{\partial D_\rho(a)}\frac{f(z)}{z-w}\,\mathrm dz\\
        &=\frac{f(z)}{2\pi i}\oint_{\partial D_\rho(a)}\sum_{n=0}^\infty\frac{(w-a)^n}{(z-a)^{n+1}}\,\mathrm dz\\
        &=\sum_{n=0}^\infty(w-a)^n\frac{1}{2\pi i}\oint_{\partial D_\rho(a)}\frac{f(z)}{(z-a)^{n+1}}\,\mathrm dz
    \end{align*}
    We can change the order of integration since a geometrical series is uniform.
    This gives us the desired series.
\end{proof}
\begin{corollary}
    If $f:U\to\mathbb C$ is holomorphic where $U$ is open, then $f$ has derivative of all orders and they are all holomorphic.
\end{corollary}
\begin{proof}
    Follows directly.
\end{proof}
A function, real or complex, is said to be analytic on an open set if it has a convergent power series representation there.
So for a complex function, being analytic is equivalent to being holomorphic.
But for real functions, even infinite differentiability does not imply a function is analytic, for example
$$f(x)=\begin{cases}
    e^{-1/x^2}\text{, for $x\neq 0$.}\\
    0\text{, for $x=0$.}
\end{cases}$$
From now on, we can use analytic and holomorphic interchangably.
We can also say now that if we decompose $f=u+iv$, then $u,v$ are automatically smooth given $f$ holomorphic.
\begin{theorem}[Morera's Theorem]
    Let a complex function $f:U\to\mathbb C$ be continuous.
    Suppose that its integral is $0$ along any closed curve, then $f$ is holomorphic.
\end{theorem}
\begin{proof}
    The antiderivative exists and is holomorphic, so its derivative $f$ is holomorphic.
\end{proof}
\begin{theorem}[Cauchy's Integral Formula for Derivatives]
    For a holomorphic function $f:D=D_r(a)\to\mathbb C$ and for any $w\in D$ and $|w|<\rho<r$, we have
    $$f^\prime(w)=\frac{1}{2\pi i}\oint_{\partial D_\rho(a)}\frac{f(z)}{(z-w)^2}\,\mathrm dz$$
    More generally,
    $$f^{(n)}(w)=\frac{n!}{2\pi i}\oint_{\partial D_\rho(a)}\frac{f(z)}{(z-w)^{n+1}}\,\mathrm dz$$
\end{theorem}
\begin{proof}
    Consider the function $g(z)=f(z)/(z-w)$ and its derivative gives the $n=1$ case.
    The general case follows from induction.
\end{proof}
\begin{definition}
    Let $U\subset\mathbb C$ be an open set and let $(f_n)$ be a sequence of complex functions.
    We say it converges locally uniformly if for any $a\in U$, there is some $r>0$ such that $(f_n)$ converges uniformly on $D_r(a)\subset U$.
\end{definition}
\begin{proposition}
    $(f_n)$ converges locally uniformly on $U$ iff it converges uniformly on any compact subset of $U$.
\end{proposition}
\begin{proof}
    Easy.
\end{proof}
\begin{theorem}
    Let $f_n:U\to\mathbb C$ be a sequence of holomorphic functions that converges to $f$ locally uniformly on $U$, then $f$ is holomorphic.
    Moreover, $f_n^\prime\to f^\prime$ locally uniformly on $U$.
\end{theorem}
\begin{remark}
    This is totally not true in the real case since we can approximate any continuous function on a closed interval locally uniformly by polynomials (Weierstrass Approximation Theorem).
\end{remark}
\begin{proof}
    $f$ is continuous since continuity is a local property and we already know that the uniform limit of continuous functions is continuous.
    Pick any point $a\in U,r>0$ with $B_r(a)\subset U$ and $f_n\to f$ uniformly on $B_r(a)$, the by Cauchy's Theorem on star-shaped domains, we have
    $$\oint_\gamma f_n(z)\,\mathrm dz=0$$
    for any closed $\gamma$ on the disk.
    Then by uniformity
    $$\oint_\gamma f(z)\,\mathrm dz=\oint_\gamma\lim_{n\to\infty}f_n(z)\,\mathrm dz=\lim_{n\to\infty}\oint_\gamma f_n(z)\,\mathrm dz$$
    So $f$ is holomorphic on the disk.
    The derivative case follows from Cauchy Integral Formula for derivatives.
\end{proof}
\begin{theorem}[Principle of Isolated Zero]
    Let $f:D_R(w)\to\mathbb C$ be holomorphic and not identically zero, then the zeros of $f$ are isolated, that is, there exists $0<r\le R$ such that $f(z)\neq 0$ for any $0<|z-w|<r$.
\end{theorem}
\begin{proof}
    Trivial.
\end{proof}
\begin{remark}
    The zeros of a holomorphic function can have a limit point on the boundary of the domain.
    For example, $f(z)=\sin(1/z)$ on $\mathbb C^\star$.
\end{remark}
\subsection{Analytic Continuation}
By Taylor Series, we know that a holomorphic function $f$ on a disk $D_R(a)$ is completely determined by the values of the derivatives of $f$ at $a$.
Does this generalize to arbitrary domains?
\begin{theorem}
    Let $D'\subset D$ be domains and $f:D'\to\mathbb C$ be analytic.
    Then there is at most one analytic function $g:D\to\mathbb C$ such that $g|_{D'}=f$.
\end{theorem}
Such a function $g$ is called an analytic continuation of $f$ to $D$.
\begin{proof}
    Suppose there are two of the functions $g_1,g_2$ satisfying this condition, then $g_1|_{D'}=g_2|_{D'}=f$ and $h=g_1-g_2$.
    Let $D_0=\{z\in D:\exists r>0, h|_{D_r(z)}\equiv 0\}$.
    Then $D_0$ is open.
    We shall show that it is also closed, which shall finish our proof.
    But $D_1=\{z\in D:\exists n\in\mathbb N, h^{(n)}(z)\neq 0\}=D\setminus D_0$ is also open in $D$, so one of them is empty by connectedness.
    But $D'\subset D_1$ hence $D_1$ is empty, therefore $h\equiv 0\implies g_1=g_2$.
\end{proof}
Actually this is a direct consequence of the Principle of Isolated Zeros, but one want to show that this depends entirely on the existence of Taylor series.
\begin{remark}
    1. The proof depends on the result that the functions have convergent Taylor series at every point, so it is valid for real valued analytic functions.\\
    2. Analytic continuations to larger domains need not always exist.
    For example, the function
    $$f(z)=\sum_{n=0}^\infty z^{n!}$$
    is analytic on $D_1(0)$, but we will prove in example sheet that this cannot be extend to larger domains.
    $\partial D_1(0)$ is called the natural boundary of $f$.
\end{remark}
\begin{corollary}[Identity Principle]
    Let $U\subset \mathbb C$ be a domain and let $f,g:U\to\mathbb C$ be holomorphic and the set $\{z\in U:f(z)=g(z)\}$ has a non-isolated point, then $f\equiv g$.
\end{corollary}
\begin{proof}
    Immediate.
\end{proof}
\begin{corollary}[Global Maximum Principle]
    Let $U\subset\mathbb C$ be a bounded domain and let $f:\bar{U}\to\mathbb C$ be continuous and holomorphic in $U$, then $\sup|f|$ is attained on the boundary $\partial U=\bar U\setminus U^\circ$.
\end{corollary}
\begin{proof}
    Follows from the Local Maximum Principle and Identity Principle.
\end{proof}
\begin{remark}
    Some of the theorems can generalize even further to help us to solve elliptic PDEs as a generlization to Laplace's Equation.
    For example, we also have an analogue of analytic continuation and we can show that harmonic functions in any dimensions are infinitely differentiable.
    Also some analogue of the Local Maximum Modulus Principle also hold.
    So does the mean value property (but this time only for harmonic functions instead of general functions satisfying elliptic PDEs).
    In fact, if the mean value property hold, the function has to be harmonic.
\end{remark}
    \section{The Winding Number}
We want to generalize Cauchy's Theorem to non-simply-connected domains and generalize Cauchy Integral Formula to general shapes.
\begin{definition}
    Let $\gamma:[a,b]\to\mathbb C$ be a closed curve and fix $w\in\mathbb C\setminus\operatorname{Im}\gamma$.
    For each $t$, we can write $\gamma(t)=w+r(t)e^{i\theta(t)}$.
    Where $r(t)=|\gamma(t)-w|$.
    If $\gamma$ is piecewise $C^1$, so is $r$.
    If we can find a continuous $\theta(t)$ such that the equation holds for every $t$, then we define the winding number (or index) if $\gamma$ about $w$ by
    $$I(\gamma;w)=\frac{\theta(b)-\theta(a)}{2\pi}$$
\end{definition}
Note that the winding number must be an integer.
It is easy to show that it is well-defined (i.e. independent of the choice of continuous $\theta$), and also hopefully we can show that it always exists.
\begin{lemma}
    If $\gamma$ is a piecewise $C^1$ curve and $w$ not in the image of $\gamma$, then there exists continuous piecewise $C^1$ real function $\theta$ such that
    $$\gamma(t)=w+r(t)e^{i\theta(t)}$$
    where $r(t)=|\gamma(t)-w|$.
\end{lemma}
If $\gamma$ is $C^1$ and the lemma holds, then $\gamma^\prime(t)=r^\prime(t)e^{i\theta(t)}+ir(t)\theta^\prime(r)e^{i\theta(t)}$, which rearranges to give
$$\theta^\prime(t)=\operatorname{Im}\left( \frac{\gamma^\prime(t)}{\gamma(t)-w} \right)\implies\theta(t)=\theta(a)+\operatorname{Im}\left( \int_a^t \frac{\gamma^\prime(x)}{\gamma(x)-w}\,\mathrm dx\right)$$
\begin{proof}
    Let $h(t)=\int_a^t\frac{\gamma^\prime(s)}{\gamma(s)-w}\,\mathrm ds$ where the singularities are skipped when integrate.
    So $h$ is continuous and differentiable wherever $\gamma$ is, where we will have $h^\prime(t)=\gamma^\prime(t)/(\gamma(t)-w)$.
    Except the singularities, we have
    $$\frac{\mathrm d}{\mathrm dt}\left( (\gamma(t)-w)e^{-h(t)} \right)=\gamma^\prime(t)e^{-h(t)}-(\gamma(t)-w)e^{-h(t)}h^\prime(t)=0$$
    So $(\gamma(t)-w)e^{-h(t)}$ is piecewise constant, but it is also continuous, so it is constant.
    $h(a)=0$, so if we let $\alpha$ be the argument of $\gamma(a)-w$, then
    \begin{align*}
        \gamma(t)-w&=(\gamma(a)-w)e^{h(t)}\\
        &=|\gamma(a)-w|e^{i\alpha}e^{\operatorname{Re}h(t)}e^{i\operatorname{Im}h(t)}\\
        &=|\gamma(a)-w|e^{\operatorname{Re}h(t)}e^{i(\alpha+\operatorname{Im}h(t))}
    \end{align*}
    Taking $\theta(t)=\alpha+\operatorname{Im}h(t)$ finishes the proof.
\end{proof}
\begin{corollary}
    If additionally $\gamma(t)$ is a closed curve, then
    $$I(\gamma;w)=\frac{1}{2\pi i}\int_\gamma\frac{\mathrm dz}{z-w}$$
\end{corollary}
\begin{proof}
    Follows directly from the above choice of $\theta$.
\end{proof}
\begin{remark}
    Actually the lemma holds for continuous $\gamma$ (where we want $r,\theta$ to be continuous).
    The proof is exercise.
\end{remark}
\begin{proposition}
    If $\gamma:[a,b]\to D_R(\alpha)$ be a piecewise $C^1$ closed curve and $w\notin D_R(\alpha)$, then $I(\gamma,w)=0$.
\end{proposition}
\begin{proof}
    Write out definition and use Cauchy.
\end{proof}
\begin{proposition}
    The function $w\mapsto I(\gamma,w)$ is locally constant.
\end{proposition}
\begin{proof}
    Immediate from continuity.
\end{proof}
\begin{definition}
    Let $U\subset\mathbb C$ is open.\\
    1. A closed curve $\gamma$ in $U$ is homologous to $0$ if $I(\gamma,w)$ is $0$ for all $w\notin U$.\\
    2. $U$ is simply connected if every closed curve in $U$ is homologous to $0$.
\end{definition}
So a disk is simply connected but an annulus is not.
\begin{theorem}[Cauchy Integral Formula]\label{general_cif}
    Let $U$ be open and let $\gamma:[a,b]\to U$ be closed and homologous to $0$, then for any holomorphic $f:U\to\mathbb C$ we have
    $$\frac{1}{2\pi i}\oint_\gamma\frac{f(z)}{z-w}\,\mathrm dz=I(\gamma,w)f(w)$$
    for any $w\in U$ that is not in the image of $\gamma$.
\end{theorem}
Note that we can assume $U$ is a bounded domain.
In particular, by integrating $F(z)=(z-w)f(z)$, we know from the theorem that
$$\int_\gamma f(z)\,\mathrm dz=0$$
\begin{corollary}
    If $U$ is simply connected, then for any closed curve $\gamma$ in $U$ and any holomorphic function $f$ on $U$, we have
    $$\int_\gamma f(z)\,\mathrm dz=0$$
\end{corollary}
\begin{proof}
    Immediate from the preceding theorem.
\end{proof}
\begin{remark}
    By the theorem, fix $\gamma$, if
    $$\oint_\gamma f(z)\,\mathrm dz=0$$
    holds for functions in the form $f(z)=1/(z-w)$, then it holds for every holomorphic $f$.
\end{remark}
\begin{proposition}
    Suppose $U\subset\mathbb C$ is open and $\phi:U\times [a,b]\to\mathbb C$ is continuous and each $z\mapsto \phi(z,s)$ is holomorphic, then
    $$\int_a^b\phi(z,s)\,\mathrm ds$$
    is holomorphic.
\end{proposition}
\begin{proof}
    Morera's and Fubini's on closed intervals.
\end{proof}
\begin{proof}[Proof of Theorem \ref{general_cif}]
    First define a function $g$ on $U\times U\to\mathbb C$ by
    $$g(z,w)=\begin{cases}
        (f(z)-f(w))/(z-w)\text{, if $z\neq w$}\\
        f^\prime(z)\text{, if $z=w$}
    \end{cases}$$
    which is continuous since $f$ is holomorphic.
    Also define
    $$h(w)=\oint_\gamma g(z,w)\,\mathrm dz,w\in U;h_1(w)=\oint_\gamma\frac{f(z)}{z-w}\,\mathrm dz,w\in\mathbb C\setminus\gamma([a,b])=U_1$$
    Also, $h=h_1$ on $U\cap U_1$ and $U\cup U_1=\mathbb C$ since $\gamma$ is homologous to zero, so we can define
    $$\phi(w)=\begin{cases}
        h(w)\text{, if $w\in U$}\\
        h_1(w)\text{, if $w\in U_1$}
    \end{cases}$$
    Then $\phi$ is entire by the preceding proposition, and since the behaviour of $\phi$ as $|w|\to\infty$ shall be that of $h_1$, and as $I(\gamma;w)=0$ for large enough $|w|$, we have
    $$|\phi(w)|\le\frac{\operatorname{length}(\gamma)\sup_\gamma|f|}{|w|-R}\to 0$$
    as $|w|\to\infty$.
    By Theorem \ref{holo_bdd_const}, $\phi$ is constantly zero, in particular, $h$ is constantly zero on $U$.
    This implies the result.
\end{proof}
The clever part of the proof that we used a global theorem (i.e. Liouville) to show a local result, which is pretty cool.
\begin{definition}
    Let $\gamma_0,\gamma_1:[a,b]\to\mathbb C$ be two closed piecewise $C^1$ curves.
    We say $\gamma_0$ is homotopic to $\gamma_1$ if there is a function $H:[a,b]\times [0,1]\to\mathbb C$ with $H(0,t)=\gamma_0(t),H(1,t)=\gamma_1(t)$, and for each $s\in [0,1]$, $H(t,s)$ is a closed piecewise $C^1$ curve, that is $\forall s\in[0,1],H(s,a)=H(s,b)$.
\end{definition}
\begin{theorem}\label{homotopy}
    For curves $\gamma_0,\gamma_1:[0,1]\to U$ be homotopic and $w\neq U$, we have $I(\gamma_0,w)=I(\gamma_1,w)$.
\end{theorem}
\begin{lemma}
    If $\gamma_0,\gamma_1:[0,1]\to\mathbb C$ are piecewise $C^1$, $w\in\mathbb C$ and
    $$\forall t\in [0,1],|\gamma_0(t)-\gamma_1(t)|<|w-\gamma_1(t)|$$
    then $I(\gamma_0;w)=I(\gamma_1;w)$.
\end{lemma}
\begin{proof}
    Exercise.
\end{proof}
We will show the $C^1$ case, but the general case is also true by similar idea.
\begin{proof}[Proof of Theorem \ref{homotopy}]
    Consider the continuous deformation $H:[0,1]\times [0,1]\to\mathbb C$.
    Now $[0,1]^2$ is compact, so $H$ has compact (hence closed) image and is uniformly continuous.
    So for $w\notin U$, so there is $\epsilon>0$, for any $(s,t)\in [0,1]^2$, $|H(s,t)-w|>2\epsilon$.
    Also there is $n\in\mathbb N$ such that
    $$|s-s'|+|t-t'|\le\frac{1}{n}\implies |H(s,t)-H(s',t')|<\epsilon$$
    Denote $H(s,t)$ by $\gamma_s(t)$, this would mean that for $k=1,2,\ldots,n$, we have $|\gamma_{(k-1)/n}(t)-\gamma_{k/n}(t)|<\epsilon$, but we also have $|w-\gamma_{k/n}(t)|>2\epsilon$, so
    $$|\gamma_{(k-1)/n}(t)-\gamma_{k/n}(t)|<\epsilon<2\epsilon<|w-\gamma_{k/n}(t)|$$
    So by the preceding lemma $I(\gamma_0;w)=I(\gamma_{1/n};w)=\cdots=I(\gamma_1;w)$.
\end{proof}
So a star-shaped domain is simply connected.
Given $\gamma:[0,1]\to U$ and let $p$ be the centre of $U$, then we can shrink $\gamma$ to $p$ in the obvious way.
\begin{remark}
    If a curve is null-homotopic, then by the theorem it is homologous to $0$, but the converse might not be true.
    The usual algebraic topological definition of simple-connectedness is that every closed curve is null-homotopic.
    So $U$ is simply-connected in the algebraic topological way implies that it is simply connected in the complex analysis way in terms of winding number.
    This then implies that Cauchy's Theorem (in simply connected domains) holds.\\
    The reverse implication holds but complex simply-connectedness implying topological simply-connectedness part is not trivial.
\end{remark}
    \section{Laurent Series, Isolated Singularities and the Residue Theorem}
\begin{theorem}
    Consider an open annulus $A=\{z\in\mathbb C:r<|z-a|<R\}$ and let $f$ be holomorphic on $A$.
    Then $f$ has a Laurent series expansion
    $$f(z)=\sum_{n\in\mathbb Z}c_n(z-a)^n=\left( \sum_{n=0}^\infty c_n(z-a)^n \right)+\left( \sum_{n=1}^\infty c_{-n}(z-a)^{-n} \right)$$
    for every $z\in A$.
    In addition, for any $r<\rho<R$, we have
    $$c_n=\frac{1}{2\pi i}\oint_{\partial D_\rho(a)}\frac{f(z)}{(z-a)^{n+1}}\,\mathrm dz$$
    Also for any $r<\rho_1<\rho_2<R$ the series converges uniformly on $\{z\in\mathbb C:\rho_1<|z-a|<\rho_2\}$
\end{theorem}
\begin{proof}
    Let $w\in A$, and choose $r<\rho_2<|w-a|<\rho_1<R$ and let $\gamma_1$ be the anticlockwise curve as the boundary of the a polar slice of the annulus $\{z\in\mathbb C:\rho_2<|z-a|<\rho_1\}$ that contains $w$, and $\gamma_2$ be the boundary of the rest such that $\gamma_{1,2}$ agrees on the slices.
    Now
    $$\oint_{\gamma_2}\frac{f(z)}{z-w}\,\mathrm dz=0,\frac{1}{2\pi i}\oint_{\gamma_1}\frac{f(z)}{z-w}\,\mathrm dz=I(\gamma,w)f(w)$$
    But since $\gamma_1$ is homotopic to a circle around $w$, $I(\gamma,w)=1$.
    Hence
    \begin{align*}
        f(w)&=\frac{1}{2\pi i}\left( \oint_{\gamma_1}\frac{f(z)}{z-w}+\oint_{\gamma_2}\frac{f(z)}{z-w} \right)\\
        &=\frac{1}{2\pi i}\left( \oint_{\partial D_{\rho_1}(a)}\frac{f(z)}{z-w}\,\mathrm dz-\oint_{\partial D_{\rho_2}(a)}\frac{f(z)}{z-w}\,\mathrm dz \right)\\
        &=f_1(w)+f_2(w)\\
        f_1(w)&=\frac{1}{2\pi i}\oint_{\partial D_{\rho_1}(a)}\frac{f(z)}{z-w}\,\mathrm dz\\
        f_2(w)&=-\frac{1}{2\pi i}\oint_{\partial D_{\rho_2}(a)}\frac{f(z)}{z-w}\,\mathrm dz
    \end{align*}
    Expanding $f_1$ as Taylor series about $a$ would give the nonnegative terms of the Laurent series.
    We shall produce the negative terms from $f_2$ by a trick we used before.
    Observe
    $$-\frac{1}{z-w}=\frac{1}{w-a}\frac{1}{1-(z-a)/(w-a)}=\sum_{m=1}^\infty\frac{(z-a)^{m-1}}{(w-a)^m}$$
    which converges uniformly as a geometric series.
    So we can change the order of integration and get
    $$f_2(w)=\sum_{m=1}^\infty\left(\frac{1}{2\pi i}\oint_{\partial D_{\rho_2}(a)}f(z)(z-a)^{m-1}\,\mathrm dz\right)(w-a)^{-m}$$
    writing $m=-n$ gives the existence of the Laurent series.
    The rest is trivial.
\end{proof}
\begin{definition}
    A complex valued function $f$ has an isolated singularity at a point $a\in\mathbb C$ if $f$ is defined and holomorphic in a punctured disk $D_r(a)\setminus\{a\}$ for some $r>0$ but not in $D_r(a)$ (i.e. either not defined at $a$ or not holomorphic there).
\end{definition}
\begin{example}
    1. $f(z)=1/z$ has an isolated singularity at $z=0$.\\
    2. $f(z)=(e^z-1)/z$ also has one at $0$.\\
    3. $f(z)=e^{1/z}$ has one at $0$.
\end{example}
\begin{definition}
    An isolated singularity $a$ of $f$ is removable if there is a holomorphic $g$ defined on $D_r(a)$ for some $r>0$ and $f=g$ on $D_r(a)\setminus\{a\}$.
\end{definition}
\begin{proposition}
    $f$ has a removable singularity at $a$ iff $\lim_{z\to a}(z-a)f(z)=0$.
\end{proposition}
\begin{proof}
    The ``only if'' part is immediate.
    For the other direction, Suppose $f$ is holomorphic on $D_r(a)\setminus\{a\}$ and $\lim_{z\to a}(z-a)f(z)=0$, then let
    $$h(z)=\begin{cases}
        (z-a)^2f(z)\text{, if $z\neq 0$}\\
        0\text{, if $z=0$}
    \end{cases}$$
    So $h$ is holomorphic in some disk $D_r(a)\setminus\{a\}$.
    Also
    $$\frac{h(z)-h(a)}{z-a}=(z-a)f(z)\to 0$$
    as $z\to a$, so $h$ is actually holomorphic on $D_r(a)$ with $h^\prime(a)=0$.
    But $h$ has a zero of order at least $2$, hence there is a holomorphic $g$ on $D_r(a)$ such that $h(z)=(z-a)^2g(z)$, but then $g$ equals $f$ on $D_r(a)\setminus\{a\}$.
\end{proof}
So the isolated singularities of bounded functions are removable.
\begin{definition}
    Let $a$ be an isolated singularity of $f:D_r(a)\to\mathbb C$, then $a$ is a pole if the limit $\lim_{z\to a}|f(z)|=\infty$.\\
    If $a$ is neither a pole nor a removable singularity, we say $a$ is an essential singularity.
\end{definition}
\begin{proposition}\label{pole_equiv}
    Let $U$ be a domain and $f:U\setminus\{a\}\to\mathbb C$ is holomorphic for $a\neq U$.
    Then the followings are equivalent:\\
    1. $a$ is a pole of $f$.\\
    2. There is $\epsilon>0$ such that there is a holomophic $h:D_\epsilon(a)\to\mathbb C$ with $h(z)=0\iff z=a$ and $\forall z\in D_\epsilon(a)\setminus\{a\},f(z)=1/h(z)$.\\
    3. There is a holomorphic $g:U\to\mathbb C$ such that $g(a)\neq 0$ and
    $$f(z)=(z-a)^{-k}g(z)$$
    for some integer $k\ge 1$.
    Also such $g,k$ are uniquely determined by $f$.
\end{proposition}
\begin{proof}
    $1\implies 2$: Choose $\epsilon$ such that $|f(z)|\ge 1$ whenever $0<|z-a|<\epsilon$, so $1/f$ is holomorphic on $D_\epsilon(a)\setminus\{a\}$.
    But $a$ becomes a removable singularity of $1/f$ by the preceding proposition, so the extension of $1/f$ to the entire disk $D_\epsilon(a)$ would be the desired $h$.\\
    $2\implies 3$: Suppose $h$ has a zero of order $k$, then $h(z)=(z-a)^kq(z)$ where $q$ is holomorphic and $q(a)\neq 0$, so $q$ is nonzero in a disk $D\subset D_\epsilon(a)$ around $a$, hence
    $$g(z)=\begin{cases}
        1/q(z)\text{, if $z\in D$}\\
        (z-a)^kf(z)\text{, if $z\in U\setminus\{a\}$}
    \end{cases}$$
    which is well-defined by the definition of $h$ and is holomorphic.
    Uniqueness follows.\\
    $3\implies 1$ is trivial.
\end{proof}
\begin{corollary}
    If $z\in\mathbb C$ is an essential singularity of $f$, then the limit of $|f(z)|$ as $z\to a$ does not exist, either as a real number or infinity.
\end{corollary}
\begin{example}
    If we compute $e^{1/z}$ as $z\to 0$, we will find that it does not exist, so $0$ is an essential singularity of it.
\end{example}
\begin{definition}
    If $f$ has a pole at $a$, then the integer $k$ as stated in Proposition \ref{pole_equiv} is called the order of the pole $a$.
    If $k=1$, then we call $a$ a simple pole.\\
    Let $U$ be a domain and $S\subset U$ is a set of isolated points in $U$.
    If $f$ is holomorphic on $U\setminus S$ and each $s\in S$ is either a removable singularity or a pole of $f$, we say $f$ is meromorphic on $U$.
\end{definition}
Meromorphic functions can be taken (maybe) as holomorphic functions from $U$ to $\mathbb C_\infty$.
\begin{remark}
    By the preceding corollary, at an isolated essential singularity $a$ of $f$, then $f$ oscillates around $a$.
    More precisely we have the Cosorati-Weierstrass Theorem (proven in example sheet), where we find that the image of any punctured neighbourhood around $a$ has dense image in $\mathbb C$.
    A much much harder theorem will show that this image is the entire complex plane $\mathbb C$ except possibly one point.
\end{remark}
\begin{proposition}
    Let $f$ be holomorphic in $D_R(a)\setminus\{a\}$, then we have the series
    $$f(z)=\sum_{n\in\mathbb Z}c_n(z-a)^n,\forall z\in D_R(a)\setminus\{a\}$$
    Also, $\forall n<0,c_n=0$ implies that $a$ is removable.\\
    If now $c_{-k}\neq 0$ for some $k>0$ and $c_n=0$ for $n<-k$, then there is a holomorphic $g:D_R(a)\to\mathbb C$ such that
    $$f(z)=\sum_{n=-k}^\infty c_n(z-a)^n=\frac{g(z)}{(z-a)^k}$$
    So $f$ has a pole of order $k$ at $a$.
\end{proposition}
\begin{proof}
    Trivial.
\end{proof}
Let rhe series expansion be as above, then by uniform convergence, the integral
$$\int_{\partial D_\rho(a)}f(z)\,\mathrm dz=2\pi ic_{-1}$$
So we define
\begin{definition}
    $c_{-1}$ is defined as the residue $\operatorname{Res}_f(a)$ of $f$ at $a$.
    And the series
    $$\sum_{n=-\infty}^{-1}c_n(z-a)^n$$
    is called the principal part of $f$.
\end{definition}
\begin{theorem}[Residue Theorem]
    Let $U$ be a domain and $S=\{a_1,\ldots,a_k\}\subset U$.
    Suppose $f$ is holomorphic on $U\setminus S$, then for any closed piecewise $C^1$ curve $\gamma:[0,1]\to U\setminus S$ homologous to $0$ in $U$, we have
    $$\int_\gamma f(z)\,\mathrm dz=2\pi i\sum_{j=1}^kI(\gamma;a_j)\operatorname{Res}_f(a_j)$$
\end{theorem}
\begin{proof}
    Let $g$ be the principal part of $f$ at $a_i$.
    Then $g$ is holomorphic on $\mathbb C\setminus\{a_i\}$, so $f-(g_1+\cdots+g_k)$ is holomorphic in $U$ except at removable singularities, so by Cauchy's Theorem,
    $$\int_\gamma f(z)\,\mathrm dz=\sum_{j=1}^k\int_\gamma g_j(z)\,\mathrm dz=2\pi i\sum_{j=1}I(\gamma,a_j)\operatorname{Res}_f(a)$$
    as desired.
\end{proof}
There are some useful facts about residues.
\begin{proposition}
    If $f$ has a simple pole at $a$, then $\operatorname{Res}_f(a)=\lim_{z\to a}(z-a)f(a)$.
    More generally, if $f$ has a pole of order $k$, then if we write $f(z)=(z-a)^{-k}g(z)$ for a holomorphic $g$, then $\operatorname{Res}_f(a)=g^{(k-1)}(a)/(k-1)!$.
    If $f=g/h$ with $g,h$ holomorphic and at $a$, $g(a)\neq 0$ and $h$ has a simple zero at $a$, then $f(a)=g(a)/h^\prime(a)$.
\end{proposition}
\begin{proof}
    Easy.
\end{proof}
\begin{proposition}[Jordan's Lemma]
    If $f$ is holomorphic in $\{z\in\mathbb C:|z|>r\}$ for some $r>0$ and if $zf(z)$ is bounded for large $|z|$, then
    $$\int_{\gamma_R}f(z)e^{i\alpha z}\,\mathrm dz\to 0,\gamma_R:[0,\pi]\ni t\mapsto Re^{it}$$
    as $R\to\infty$.
\end{proposition}
\begin{proof}
    Example sheet.
    Use the fact that $\sin t/t\ge 2/\pi$ for $t\in (0,\pi/2]$.
\end{proof}
\begin{proposition}
    Let $f$ be holomorphic on $D_R(a)\setminus\{a\}$ with a simple pole at $z=a$ and $\gamma_\epsilon:[\alpha,\beta]\ni t\mapsto a+\epsilon e^{it}$, then
    $$\lim_{\epsilon\to 0^+}\int_{\gamma_\epsilon}f(z)\,\mathrm dz=(\beta-\alpha)i\operatorname{Res}_f(a)$$
\end{proposition}
\begin{proof}
    Just write $f(z)=\operatorname{Res}_f(a)/(z-a)+g(z)$.
\end{proof}
\begin{example}
    Let $R>\epsilon>0$.
    Consider $f(z)=e^{iz}/z$ along the contour $\gamma=\gamma_1-\gamma_\epsilon+\gamma_2+\gamma_R$ where $\gamma_1$ is the segment $[-R,-\epsilon]$, $\gamma_2$ is $[\epsilon,R]$, and $\gamma_R(t)=Re^{it},\gamma_\epsilon(t)=\epsilon e^{it},t\in [0,\pi]$.
    Now by Jordan's Lemma and the preceding proposition respectively, we have
    $$\lim_{R\to\infty}\int_{\gamma_R}f(z)\,\mathrm dz=0,\lim_{\epsilon\to 0}\int_{\gamma_\epsilon}f(z)\,\mathrm dz=\pi i\operatorname{Res}_f(a)=\pi i$$
    Now for $\epsilon,R$ as before we have
    $$0=\int_\gamma f(z)\,\mathrm dz=\int_{-R}^{-\epsilon}\frac{e^{ix}}{x}\,\mathrm dx+\int_{-\gamma_\epsilon}f(z)\,\mathrm dz+\int_\epsilon^R\frac{e^{ix}}{x}\,\mathrm dx+\int_{\gamma_R}f(z)\,\mathrm dz$$
    Now let $R\to\infty,\epsilon\to 0$, we get
    $$2i\int_0^\infty\frac{\sin x}{x}\,\mathrm dx=\pi i\implies\int_0^\infty\frac{\sin x}{x}\,\mathrm dx=\frac{\pi}{2}$$
\end{example}
    \section{The Argument Principle, Local Degree and Rouch\'e's Theorem}
\begin{definition}
    Let $D$ be a domain, we say a closed curve $\gamma:[0,1]\to\mathbb C$ bounds $D$ if $I(\gamma;w)=1$ for any $w\in D$ and $I(\gamma;w)=0$ for any $w\notin D\cup\gamma([0,1])$.
\end{definition}
Note that the orientation matters here.
$D$ is indeed bounded since $\gamma([0,1])$ is contained in $D_R(0)$ for a large enough $R$, but $I(\gamma;w)=0$ for any $w\notin D_R(0)$, so $D\subset D_R(0)$, hence is bounded.
\begin{theorem}[The Argument Principle]
    Let $\gamma$ be a closed curve bounding a domain $D$ and suppose that $f$ is meromorphic in some open set $U\subset \bar{D}\cup\gamma([0,1])$ such that $f$ has no pole or zero on $\gamma([0,1])$.
    If $f$ has precisely $N$ zeros and $P$ poles in $D$ (both counting with multiplicity), then
    $$N-P=\frac{1}{2\pi i}\int_\gamma\frac{f^\prime(z)}{f(z)}\,\mathrm dz=I(\Gamma;0)$$
    where $\Gamma=f\circ\gamma$.
\end{theorem}
\begin{proof}
    $N$,$P$ are finite since $D$ is bounded and zeros and poles are isolated points.
    Also note that $0\notin\Gamma([0,1])$ since $f\circ\gamma$ is never $0$.
    So
    $$I(\Gamma,0)=\frac{1}{2\pi i}\int_\Gamma\frac{1}{z}\,\mathrm dz=\frac{1}{2\pi i}\int_0^1\frac{1}{f(\gamma(t))}f^\prime(\gamma(t))\gamma^\prime(t)\,\mathrm dt=\frac{1}{2\pi i}\int_\gamma\frac{f^\prime(z)}{f(z)}\,\mathrm dz$$
    which is the second equality.\\
    As for the other part, note that if $a$ is neither a pole or a zero at $f$, then $f^\prime/f$ is holomorphic at $a$.
    Also, if $a$ is a zero (or a pole) of order $k$ at $z=a$, then $f^\prime/f$ has a simple pole at $z=a$ with $\operatorname{Res}_{f^\prime/f}(a)=k$ (or $-k$ respectively).
    Now $I(\gamma,w)=0$ for any $w\in D$ by hypothesis, so applying the residue theorem to $f^\prime/f$ shall give the result.
\end{proof}
\begin{definition}
    Let $f:D_R(a)\to\mathbb C$ be holomorphic and nonconstant, then the local degree of $f$ at $a$, denoted $\deg_f(a)$, is the order of zero of $f(z)-f(a)$ at $0$.
\end{definition}
Note that the local degree has to be positive.
\begin{theorem}[Local Degree Theorem]
    Consider $f:D_R(a)\to\mathbb C$ nonconstant and holomorphic.
    Suppose $\deg_f(a)=d>0$, then for every sufficiently small $r>0$, there is $\epsilon>0$ such that for any $w$ with $0<|w-f(a)|<\epsilon$, the equation $f(z)=w$ has precisely $d$ roots in $D(a,r)\setminus\{a\}$ that are all distinct.
\end{theorem}
\begin{proof}
    Since $f$ is nonconstant, by the principle of isolated zeros, we can choose $r>0$ such that $f(z)-f(a)\neq 0$ and $f^\prime(z)\neq 0$ for any $z\in B_r(a)\setminus\{a\}$.
    Now take $\gamma(t)=a+re^{it}$ for $t\in [0,2\pi]$, then $f(\gamma(t))\neq f(a)$ for any $t$, so $\Gamma=f\circ\gamma$ misses $f(a)$, so we can choose $\epsilon$ such that $\Gamma$ never enters $D_\epsilon(f(a))$.
    Then for $w\in D_\epsilon(f(a))\setminus\{f(a)\}$, by the argument principle, the number of zeros counting multiplicity of $f(z)-w$ in $D_r(a)$ is $I(\Gamma;w)=I(\Gamma;f(a))=d$.
    These zeros must all be in $D_r(a)\setminus\{a\}$ and none of them has multiplicity more than one as $f^\prime(a)\neq 0$.
\end{proof}
\begin{corollary}
    A nonconstant holomorphic function on a domain is an open map.
\end{corollary}
\begin{proof}
    Follows directly.
\end{proof}
\begin{theorem}[Rouch\'e's Theorem]
    Let $\gamma:[0,1]\to\mathbb C$ be a curve that bounds a domain $D$ and $f,g$ holomorphic on a open set $U$ containing $\bar D\cup\gamma([0,1])$.
    If $|f|>|g|$, then $f,f+g$ have the same number of zeros on $D$.
\end{theorem}
\begin{proof}
    $|f|>|g|$ hence $f,f+g$ are nowhere zero on $\gamma$, so we apply the argument principle on $h=(f+g)/f=1+g/f$.
    We have $I(h\circ\gamma;0)=0$, so $h$ has $N=P$, so $f+g,f$ has the same number of zeros.
\end{proof}
\end{document}