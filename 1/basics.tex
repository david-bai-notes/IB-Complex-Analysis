\section{Basic Notions}
\subsection{Definitions}
\begin{definition}
    A subset $U\subset\mathbb C$ is open if $\forall u\in U,\exists r>0,D_r(u)\subset U$.
\end{definition}
If we identify $\mathbb C$ with $\mathbb R^2$ in the obivous way, and give $\mathbb R^2$ the usual topology, then $U\subset\mathbb C$ is open iff it is open in $\mathbb R^2$.\\
We shall be interested in functions $f:U\to\mathbb C$ where $U$ is open.
\begin{definition}
    Let $U\subset\mathbb C$ be open and $f:U\to\mathbb C$.
    We say $\lim_{z\to c}f(z)=A$ if $\forall\epsilon>0,\exists\delta>0$,
    $$0<|z-c|<\delta\implies |f(z)-f(c)|<\epsilon$$
    $f$ is continuous at $c\in U$ if $\lim_{z\to c}f(z)=f(c)$.
    $f$ is continuous in $U$ if it is continuous everywhere in $U$.
\end{definition}
We can always write $f(x+iy)=u(x,y)+iv(x,y)$ where $u,v:\mathbb U\to\mathbb R$ where $U$ is open in $\mathbb R^2$.
It is easy to see that $\mathbb C$ inherits the continuity condition in $\mathbb R^2$ since it inherits the topology from it.
So we have that the continuity of $f$ is equivalent to that of $u,v$.
\begin{definition}
    Let $f:U\to\mathbb C$ be as before, and $w\in U$.
    We say $f$ is differentiable at $w$ if the limit
    $$\lim_{z\to w}\frac{f(z)-f(w)}{z-w}$$
    exists.
    If it exists, we say $f$ is differentiable at $w$ and has derivative equals the limit.\\
    If $f$ is differentiable everywhere in an open neighbourhood of $w$, then we say $f$ is holomorphic at $w$.
    \footnote{Some authors use the word `analytic'}\\
    We say $f$ is holomorphic on $U$ if it is holomorphic everywhere on $U$.
    (Or equivalently it is differentiable everywhere.)
\end{definition}
Rules that can be obtained from real differentiation by first principle mostly extends to complex differentiation.
For example, polynomials are differentiable everywhere, and rational functions are differentiable in the subset (which is open as the complement of a finite subset) where they are defined.
\subsection{The Cauchy-Riemann Equation}
Although complex differentiation exhibits similar definition as real differentiation, they behave very differently.
A natural question is, is the differentiability of $f$ behave the same as the differentiability of $u,v$?
The answer is no.
\begin{theorem}[Cauchy-Riemann Equation]
    $f:U\to\mathbb C$ is differentiable at $w=c+id$ iff the functions $u,v$ are differentiable at $(c,d)$ and they satisfies $u_x=v_y,u_y=-v_x$.
\end{theorem}
\begin{proof}
    By definition, $f$ is differentiable at $w$ with derivative $f^\prime(w)=p+iq$ if and only if
    $$\lim_{z\to w}\frac{f(z)-f(w)-f^\prime(z)(z-w)}{|z-w|}=0$$
    which is equivalent to the case where we seperate the real and imaginary part, which is just to say
    $$
    \begin{cases}
        \lim_{(x,y)\to(c,d)}\frac{u(x,y)-u(c,d)-(p(x-c)-q(y-d))}{\|(x-c,y-d)\|}=0\\
        \lim_{(x,y)\to(c,d)}\frac{v(x,y)-v(c,d)-(p(x-c)+q(y-d))}{\|(x-c,y-d)\|}=0
    \end{cases}$$
    which happens iff $Du(c,d)=(p,-q)$ and $Dv(c,d)=(p,q)$.
    The theorem follows.
\end{proof}
\begin{remark}
    Just because $u,v$ has partial derivatives satisfying Cauchy-Riemann equation does not guarantee the total differentiability.
    A deeper question is what if we require them to be so on an open set, but that goes beyond the scope of the course.
\end{remark}
From the theorem, we have obtained an expression of the complex derivative $f^\prime=u_x+iv_x$.
If we just want to show that the differentiability of $f$ at $c+id$ implies the existence of partials of $u,v$ satisfying Cauchy-Riemann equations, then we can just proceed by taking the limit of $z-w\to 0$ from both axes.
\begin{example}
    $f(z)=\bar{z}$ is nowhere differentiable since it does not satisfy the first C-R equation.
\end{example}
\begin{remark}
    Complex differentiability is much more restrictive than real ones.
    Of course, we will exhibit examples to justify it.
\end{remark}
\begin{theorem}[Liouville Theorem]\label{holo_bdd_const}
    If $f:\mathbb C\to \mathbb C$ is holomorphic and bounded, then $f$ is constant.
\end{theorem}
\begin{theorem}\label{infinite_holo}
    Let $U$ be an open set in $\mathbb C$, then if $f:U\to\mathbb C$ is holomorphic, so is $f^\prime$.
\end{theorem}
\begin{theorem}\label{uniform_holo}
    A sequence of holomorphic functions $f_n:U\to\mathbb C$ on the same open domain, and $f_n\to f$ uniformly, then $f$ is holomorphic.
\end{theorem}
By Theorem \ref{infinite_holo}, holomorphic functions are infinite differentiable, so any order of partial derivatives of $u,v$ exists, so we can differentiate the Cauchy-Riemann equation one more time to get $u_{xx}+u_{yy}=0$, so $u$ is a harmonic function, similarly $v$ is harmonic as well.
So the real and imaginary part of a holomorphic function are harmonic.
\begin{corollary}
    Let $f=u+iv:U\to\mathbb C$, suppose that $u,v$ has continuous partial derivatives in $U$ and $u,v$ satisfy Cauchy-Riemann Equation, then $f$ is holomorphic.
\end{corollary}
\begin{proof}
    Immediate from what we have got and a result from Analysis and Topology.
\end{proof}
\begin{remark}
    Once we know Theorem \ref{infinite_holo}, the converse of the preceding corollary follows.
\end{remark}
\begin{definition}
    A curve is a continuous map $\gamma:[a,b]\to\mathbb C$.
    It is called $C^1$ if $\gamma^\prime$ exists and is continuous on $[a,b]$.\\
    An open subset $U\subset\mathbb C$ is path connected if for any $z,w\in U$, there is a curve $\gamma:[a,b]\to U$ such that $\gamma(a)=z,\gamma(b)=w$.\\
    A non-empty open path-connected subset of $\mathbb C$ is called a domain.
\end{definition}
\begin{corollary}
    Let $U$ be a domain and $f:U\to\mathbb C$ is homomorphic. If $f^\prime\equiv 0$, then $f$ is constant.
\end{corollary}
\begin{proof}
    Proved in Analysis and Topology.
\end{proof}
\subsection{Power Series}
\begin{theorem}
    For any sequence of complex numbers $(c_n)$, there is some $R\in [0,\infty]$ such that the power series
    $$\sum_{n=0}^\infty c_n(z-a)^n$$
    such that it converges absolutely for $|z-a|<R$ and diverges for $|z-a|>R$.\\
    If $R\in\mathbb R_{>0}$, and if $0<r<R$, then the convergence is uniform on $B_r(a)$ (or $D_r(a)$, they are equivalent anyways).
\end{theorem}
\begin{definition}
    Such an $R$ is called the radius of convergence.
\end{definition}
\begin{proposition}
    $R=\frac{1}{\lambda}$ where $\lambda=\limsup_{n\to\infty}\sqrt[n]{|c_n|}$.
\end{proposition}
\begin{theorem}
    Define $f$ on the disk $D_R(a)$ by
    $$f(z)=\sum_{n=0}^\infty c_n(z-a)^n$$
    where $R>0$ is the radius of convergence of the series.
    Then\\
    1. $f$ is holomorphic on this disk,\\
    2. And
    $$\sum_{n=0}^\infty (n+1)c_{n+1}(z-a)^n$$
    is convergent on $D_R(a)$ and is the derivative of $f$.\\
    3. $f$ has derivatives of all orders on $D_R(a)$ and $f^{(n)}(a)=c_nn!$.\\
    4. If $f$ vanished in some open disk $D_r(a)$ with $0<r<R$, then it vanishes on the whole of $D_R(a)$.
\end{theorem}
\begin{proof}
    3 follows from 2 and 4 follows from 3.\\
    WLOG $a=0$.
    To prove 1 and 2, let $|z|<R$ and choose $\rho$ such that $|x|<\rho<R$, then we have
    $$\lim_{n\to\infty}\frac{n|c_n||z|^{n-1}}{|c_n|\rho^n}=\lim_{n\to\infty}\frac{n}{\rho}\left( \frac{|z|}{\rho} \right)^n\to 0$$
    so the derived series also converges on $D_R(0)$.\\
    To show that $f$ is holomorphic with derivative equals the derived series, we fix some $w\in D_R(0)$.
    Note that $f$ is differentiable at $w$ with derivative being the derived series $\sigma$ if and only if the function
    $$g(h)=
    \begin{cases}
        \frac{f(z)-f(w)}{z-w}\text{, if $z\neq w$}
        \sigma\text{, if $z=w$}
    \end{cases}$$
    is continuous at $w$.
    But $g(z)=\sum_{n=0}^\infty h_n(z)$ where
    $$h_n(z)=\begin{cases}
        c_n(z^{n-1}+z^{n-2}w+\cdots+w^{n-1})\text{, for $z\neq w$}\\
        nc_nw^{n-1}\text{, for $z=w$}
    \end{cases},n>0;h_0(z)=0$$
    Note that each $h_n$ is continuous.
    So the continuity of $g$ at $w$ follows from the uniform convergence of the series.
    Again we take $0<|w|<r<R$, so for any point $z\in D_r(0)$, we have
    $$|h_n(z)|=|c_n||z^{n-1}+z^{n-2}w+\cdots+w^{n-1}|\le n|c_n||r|^{n-1}$$
    which proves the uniform convergence.
    So 1 and 2 are proved.
\end{proof}
\begin{definition}
    The exponential function is defined as
    $$e^z=\exp(z)=\sum_{k=0}^\infty\frac{z^k}{k!}$$
\end{definition}
\begin{definition}
    An entire function is a function that is holomorphic on all of $\mathbb C$.
\end{definition}
\begin{proposition}
    1. $\exp$ is entire with derivative equals to itself.\\
    2. $\exp(z+w)=\exp(z)\exp(w)$.\\
    3. $\exp(z)\neq 0$.\\
    4. $\exp(z)=1\iff z\in2\pi i\mathbb Z$.\\
    5. $\exp:\mathbb C\to\mathbb C\setminus\{0\}$ is surjective.
\end{proposition}
\begin{proof}
    Trivial.
\end{proof}
\begin{definition}
    Given $z\in\mathbb C$, we say $w\in\mathbb C$ is a logarithm of $z$ of $e^w=z$.
\end{definition}
Note that $z$ has a logarithm iff $z\neq 0$.
If $w_1,w_2$ are two logrithms of $z$, then we immediately have $w_1-w_2\in 2\pi i\mathbb Z$.
\begin{definition}
    Let $U\subset\mathbb C^\star=\mathbb C\setminus\{0\}$.
    A branch of the logarithm on $U$ is a continuous function $\lambda:U\to\mathbb C$ such that $\exp\circ\lambda=\operatorname{id}$.
\end{definition}
\begin{remark}
    1. If such $\lambda$ does exixt, then $\lambda$ is holomorphic on $U$ by simply computation.\\
    2. From the definition, it follows that $|z|=e^{\operatorname{Re}\lambda(z)}$, so it is immediate that any branch $\lambda$ has $\operatorname{Re}\lambda(z)=\log|z|$.
\end{remark}
\begin{definition}
    The principal branch of logarithm is the function $\operatorname{Log}:U_1=\mathbb C\setminus\{x\in\mathbb R:x\le 0\}\to\mathbb C$ given by $\operatorname{Log}(z)=\log|z|+i\arg z$ where the value of $\arg$ is taken in the open interval $(-\pi,\pi)$. 
\end{definition}
Obviously in this open set it is a inverse of $\exp$.
To see the continuity, observe that the map $z\mapsto z/|z|$ is continuous and the map $\theta\mapsto e^{i\theta}$ is a homeomorphism $(-\pi,\pi)\to S\setminus\{-1\}$
\begin{proposition}
    1. $\operatorname{Log}$ is holomorphic on $U_1$ with derivative $1/z$.\\
    2. For $|z|<1$,
    $$\operatorname{Log}(z)=\sum_{n=1}^\infty\frac{(-1)^{n+1}z^n}{n}$$
\end{proposition}
\begin{proof}
    Trivial.
\end{proof}
\begin{remark}
    There is no way to extend $\operatorname{Log}$ to the whole of $\mathbb C^\star$ and remain a branch since
    $$\lim_{\theta\to \pi^+}e^{i\theta}\neq\lim_{\theta\to \pi^-}e^{i\theta}$$
    In fact there is no branch that can be such extended.
\end{remark}
Using $\exp$ and $\operatorname{Log}$, we can construct the familiar functions we had in real analysis, like the trigonometrics, hypertrigonometrics, etc., in the way we know (like $\cos(z)=(e^{iz}+e^{-iz})/2$, etc).
We can also define, for $z\notin \mathbb R_{\le 0}$, $z^\alpha=\exp(\alpha\operatorname{Log}(z))$.\\
Let $f:U\to\mathbb C$ be a holomorphic function.
A number $w\in U$ having $f^\prime(w)$ is nice in the sense that $f$ is invertible there by Inverse Function Theorem with $(f^{-1})^\prime(f(w))=1/f^\prime(w)$.
One can check that $f^{-1}$ also satisfies the Cauchy-Riemann equation, hence is also holomorphic.\\
It is also nice in a geometric sense as $f$ preserves angles at that point.
Let $\gamma_1,\gamma_2:[-1,1]\to\mathbb C$ be $C^1$ curves with $\gamma_1(0)=\gamma_2(0)=w$ and $\gamma_1^\prime(0)\gamma_2^\prime(0)\neq 0$, then since $(f\circ \gamma_1)^\prime(0)=f^\prime(w)\gamma^\prime_1(0)$.
Similar for $\gamma_2$.
Since $f^\prime(w)\neq 0$, we have
$$\frac{\gamma^\prime_1(0)}{\gamma_2^\prime(0)}=\frac{(f\circ\gamma_1)^\prime(0)}{(f\circ\gamma_2)^\prime(0)}$$
So $f$ preserves angles.
\begin{definition}[Conformal Equivalence]
    Let $U$ be an open set.
    A holomorphic function $f:U\to\mathbb C$ is called conformal at $w\in U$ if $f^\prime(w)\neq 0$.\\
    If for domains $D,D'$ there is a holomorpic bijection $f:D\to\tilde{D}$ which is conformal at every point, then we say $D,D'$ are conformal equivalent.
\end{definition}
\begin{example}
    1. Any Mobius map $f:\mathbb C\cup\{\infty\}\to\mathbb C\cup\{\infty\}$ is a conformal equivalence.\\
    2. The map $z\mapsto z^n$ from $\{\mathbb z\in\mathbb C^\star:0\le\arg z\le \pi/n\}$ to the upper half plane $\mathbb H=\{z\in\mathbb C:\operatorname{Im}z>0\}$ is conformal.\\
    3. The exponential $\exp:\{z\in\mathbb C:-\pi<\operatorname{Im}(z)<\pi\}\to\mathbb C\setminus\mathbb R_{\le 0}$.\\
    4. Consider $g:z\mapsto (z-i)/(z+i)$ with $g:\mathbb H\to D_1(0)$.
    It is also conformal.
\end{example}
\begin{theorem}[Riemann Mapping]
    Let $D$ be a domain bounded by a simple closed curve, then it is conformally equivalent to the unit disk.\\
    More generally, any simply connected domain that is not the whole complex plane is conformally equivalent to the unit disk.
\end{theorem}