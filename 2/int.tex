\section{Complex Integration}
\subsection{Definition}
We want to generalize the notion of the real Riemann integration to the integration to complex valued functions on the complex plane.
\begin{definition}
    If $f:[a,b]\to\mathbb C$ is continuous, then we define the integral of $f$ to be
    $$\int_a^b f(x)\,\mathrm dx :=\int_a^b\operatorname{Re}f(x)\,\mathrm dx+i\int_a^b\operatorname{Im}f(x)\,\mathrm dx$$
\end{definition}
Easy to check that both integrals are well-defined and the integral is linear.
\begin{proposition}
    $$\left|\int_a^bf(t)\,\mathrm dt\right|\le (b-a)\sup_{t\in [a,b]}|f(t)|$$
\end{proposition}
\begin{proof}
    If the integral is zero then there is nothing to prove.
    Otherwise we can write it in the form $re^{i\theta}$ for some $r\in\mathbb R_{>0},\theta\in\mathbb R$, so
    \begin{align*}
        \left|\int_a^bf(t)\,\mathrm dt\right|=r&=\int_a^be^{-i\theta}f(t)\,\mathrm dt\\
        &=\int_a^b\operatorname{Re}(e^{-i\theta}f(t))\,\mathrm dt\\
        &\le \int_a^b|\operatorname{Re}(e^{-i\theta}f(t))|\,\mathrm dt\\
        &\le \int_a^b|f(t)|\,\mathrm dt\\
        &\le (b-a)\sup_{t\in [a,b]}|f(t)|
    \end{align*}
    As desired.
\end{proof}
Note that the equality holds iff $f$ is constant.
\begin{definition}
    Let $\gamma:[a,b]\to\mathbb C$ be a $C^1$ curve, then the length of $\gamma$ is
    $$\int_a^b|\gamma^\prime(t)|\,\mathrm dt$$
    Also this curve is called simple iff $\gamma(t_1)=\gamma(t_2)\iff t_1\equiv t_2\pmod{b-a}$
\end{definition}
\begin{definition}
    Let $\gamma:[a,b]\to\mathbb C$ be a $C^1$ curve and $f:U\to\mathbb C$ be continuous, then we define the integral of $f$ over $\gamma$ by
    $$\int_\gamma f(t)\,\mathrm dt=\int_a^b(f\circ\gamma)(t)\gamma^\prime(t)\,\mathrm dt$$
\end{definition}
One can check that
\begin{proposition}
    1.
    $$\left(\int_{\gamma}f\right)+\alpha\left(\int_\gamma g\right)=\int_\gamma(f+\alpha g)$$
    2. 
    $$\int_{\gamma\pm\delta}=\int_\gamma\pm\int_\delta$$
    3. Let $\gamma,\delta$ be two parameterizations of the same curve linked by an injective $C^1$ function, then
    $$\int_\gamma=\int_\delta$$
\end{proposition}
Where the addition and substraction of paths are defined the way a sensible person would expect.
\begin{proof}
    Trivial.
\end{proof}
\begin{definition}
    For continuous piecewise $C^1$ curves $\gamma=\gamma_1+\gamma_2+\cdots+\gamma_n$, we set
    $$\int_\gamma=\sum_{k=1}^n\int_{\gamma_k}$$
\end{definition}
Note that by additivity of the integral over paths, this is well-defined.
\begin{proposition}
    For any continuous function $f:U\to\mathbb c$ and any (piecewise $C^1$) curve $\gamma:[a,b]\to U$, we have
    $$\left|\int_\gamma f(z)\,\mathrm dz\right|\le\sup_{t\in[a,b]}|f(\gamma(t))|\int_a^b|\gamma^\prime(t)|\,\mathrm dt$$
\end{proposition}
\begin{proof}
    Suffices to show the case when $\gamma$ is $C^1$, then
    \begin{align*}
        \left|\int_\gamma f(z)\,\mathrm dz\right|&=\left|\int_a^bf(\gamma(t))\gamma^\prime(t)\,\mathrm dt\right|\\
        &\le\int_a^b|f(\gamma(t))||\gamma^\prime(t)|\,\mathrm dt\\
        &\le\sup_{t\in[a,b]}|f(\gamma(t))|\int_a^b|\gamma^\prime(t)|\,\mathrm dt
    \end{align*}
    As desired.
\end{proof}
\subsection{Cauchy's Theorem}
\begin{theorem}[Fundamental Theorem of Calculus for Complex Integrals]
    For a continuous function $f:U\to\mathbb C$, if there is a holomorphic $F:U\to\mathbb C$ such that $F^\prime=f$, then for any (piecewise $C^1$) curve $\gamma:[a,b]\to\mathbb C$ we have
    $$\int_\gamma f(z)\,\mathrm dz=F(\gamma(b))-F(\gamma(a))$$
\end{theorem}
In particular we have
$$\oint_\gamma f(z)\,\mathrm dz=0$$
For closed $\gamma$.
\begin{proof}
    Again suffices to consider $\gamma$ as $C^1$, then we have
    $$\int_\gamma f(z)\,\mathrm dz=\int_a^bF^\prime(\gamma(t))\gamma^\prime(t)\,\mathrm dt=\int_a^b(F\circ\gamma)^\prime(t)\,\mathrm dt=F(\gamma(b))-F(\gamma(a))$$
    Done.
\end{proof}
\begin{example}
    For $\gamma:[0,\pi]\to\mathbb C$ by $t\mapsto Re^{2it}$, we have for $n\neq -1$
    $$\oint_\gamma z^n\,\mathrm dz=0$$
    Indeed it is the derivative of the holomorphic function $z^{n+1}/(n+1)$.\\
    But for $n=-1$,
    $$\oint_\gamma z^{-1}\,\mathrm dz=2\pi R$$
    which is nonzero, hence it does not have an antiderivative defined on any open set containing $\gamma$, so logarithm has no branch on $\mathbb C^\star$.
\end{example}
What is interesting is the converse of the theorem.
\begin{theorem}
    Let $U$ be a path-connected open set, and $f:U\to\mathbb C$ be continuous.
    If for any (piecewise $C^1$) closed curve $\gamma:[a,b]\to U$ we have
    $$\oint_\gamma f(z)\,\mathrm dz$$
    then $f$ has an antiderivative on $U$.
\end{theorem}
\begin{proof}
    Fix $\alpha\in U$.
    Consider the function $F:U\to\mathbb C$ with
    $$F(z)=\int_\gamma f(z)\,\mathrm dz$$
    where $\gamma$ is a piecewise $C^1$ curve on $U$ connecting $\alpha$ and $z$.
    To see the existence of $\gamma$, we know by path-connectedness of $U$ that there is a continuous curve from $\alpha$ to $z$, then we can construct a piecewise $C^1$ one by a compactness argument.\\
    Then such an $F$ is well defined by our condition.
    One can simply check to see that $F$ is holomorphic and $F^\prime=f$.
\end{proof}
\begin{definition}
    A domain $U$ is star-shaped if $\exists s\in U$ such that any other $x\in U$, there is a straight line joining $x$ and $s$.
\end{definition}
\begin{definition}
    A triangle $T$ is the convex hull of three non-colinear points on the complex plane, so
    $$T(z_1,z_2,z_3)=\{az_1+bz_2+cz_3:a,b,c\in [0,1],a+b+c=1\}$$
    We denote by $\partial T$ the boundary of $T$, which is the union of three line segments, and we choose it to be with anticlockwise direction. 
\end{definition}
\begin{corollary}
    In any star-shaped domain, if for any triangle $T$ in the domain we have
    $$\oint_{\partial T}f(z)\,\mathrm dz=0$$
    then $f$ admits a holomorphic antiderivative.
\end{corollary}
\begin{proof}
    Basically the same proof but take $\alpha=s$ and take the path to be a straight line.
\end{proof}
\begin{theorem}[Cauchy's Theorem]
    For a holomorphic $f$ and closed $\gamma$,
    $$\oint_\gamma f(z)\,\mathrm dz=0$$
\end{theorem}
\begin{theorem}[Cauchy's Theorem for Triangles]
    Let $U\subset C$ be open and $f:U\to\mathbb C$ holomorphic.
    If $T$ is a triangle in $U$, then
    $$\oint_{\partial T}f(z)\,\mathrm dz=0$$
\end{theorem}
\begin{proof}
    We subdivide the triangles by joining the midpoints, so we disassembles $T$ into $4$ smaller triangles.
    Call them $T^1,T^2,T^3,T^4$, and the directions of their boundaries are consistently given (anticlockwise), so we have
    $$\oint_{\partial T}f(z)\,\mathrm dz=\sum_{k=1}^4\oint_{\partial T^k}f(z)\,\mathrm dz$$
    We set
    $$\eta(T)=\oint_{\partial T}f(z)\,\mathrm dz$$
    So $\eta(T)=\sum_{k}\eta(T^k)$, hence there is some $k$ such that $|\eta(T^k)|\ge|\eta(T)|/4$, and $\operatorname{length}(\partial T^k)=\operatorname{length}(\partial T)/2$, so we repeat this process to get a nested sequence of triangles $T=T_0\supset T_1\supset T_2\supset\cdots$ such that $|\eta(T_k)|/4\le|\eta(T_{k+1})|$ and $\operatorname{length}(\partial T_{k+1})=\operatorname{length}(\partial T_k)/2$.
    But each $T_k$ is closed and the diameter goes to $0$, hence $\bigcap_kT_k=\{z_0\}$ for some $z_0\in\mathbb C$.
    For any $\epsilon>0$, there is some $\delta>0$ such that $|z-z_0|<\delta\implies |f(z)-f(z_0)-f^\prime(z_0)(z-z_0)|<\epsilon|z-z_0|$
    For $n$ large enough, we have $T_n\subset D_\delta(z_0)$.
    \begin{align*}
        |\eta(T_n)|&=\left|\oint_{\partial T_n}f(z)-(f(z_0)+f^\prime(z_0)(z-z_0))\,\mathrm dz\right|\\
        &\le\sup_{z\in \delta T_n}|f(z)-(f(z)+f^\prime(z_0)(z-z_0)|\operatorname{length}(\partial T_n)\\
        &\le\epsilon\sup_{z\in\partial T_n}|z-z_0|\operatorname{length}(\partial T_n)\\
        &\le\epsilon(\operatorname{length}(\partial T_n))^2
    \end{align*}
    So
    $$\frac{\eta(T)}{4^n}\le\epsilon(\operatorname{length}(\partial T^n))^2=\frac{\epsilon(\operatorname{length}(\partial T))^2}{4^n}\implies\forall\epsilon>0,\eta(T)\le \epsilon(\operatorname{length}(\partial T))^2$$
    Hence we must have $\eta(T)=0$.
\end{proof}
\begin{theorem}
    Let $f:U\to\mathbb C$ be continuous.
    If $S\subset U$ is a finite set and if $f$ is holomorphic in $U\setminus S$, then
    $$\oint_{\partial T}f(z)\,\mathrm dz=0$$
    for any triangle $T\subset U$.
\end{theorem}
\begin{proof}
    Subdivide $T$ into $N=4^n$ parts as before to $T_1,T_2,\ldots,T_N$, then $|I|\le 6|S|$ where $I=\{j:T_j\cap S\neq\varnothing\}$.
    Hence we have, by Cauchy's Theorem on triangles,
    \begin{align*}
        \left|\oint_{\partial T}f(z)\,\mathrm dz\right|&=\left|\sum_{j\in I}\oint_{\partial T_j}f(z)\,\mathrm dz\right|\\
        &\le\sum_{j\in I}\sup_{z\in\partial T_j}|f(z)|\operatorname{length}(\partial T_j)\\
        &\le 6|S|\sup_{z\in\partial T}|f(z)|\operatorname{length}(\partial T)\frac{1}{2^n}
    \end{align*}
    Letting $n\to\infty$ finishes the proof.
\end{proof}
\begin{corollary}[Cauchy's Theorem on Star-Shaped Domains]
    Let $U\subset C$ be a star-shaped domain and $f:U\to\mathbb C$ be continuous, and holomorphic on $U\setminus S$ where $S$ is a finite set.
    Then
    $$\oint_\gamma f(z)\,\mathrm dz=0$$
    for any closed curve on $U$.
\end{corollary}
\begin{proof}
    Follows directly.
\end{proof}
\subsection{Cauchy Integral Formula and Consequences}
\begin{theorem}[Cauchy's Integral Formula for a disk]
    Let $D=D_r(a)$, and let $f:D\to\mathbb C$ be holomorphic, then for any $0<\rho<r$ and any $w\in D_\rho(a)$, we have
    $$f(w)=\frac{1}{2\pi i}\oint_{\partial D_\rho(a)}\frac{f(z)}{z-w}\,\mathrm dz$$
    where $\partial D_\rho(a)$ denotes the curve $[0,1]\ni t\mapsto a+\rho e^{2\pi it}$
\end{theorem}
In particular,
$$f(a)=\int_0^1 f(a+\rho e^{2\pi it})\,\mathrm dt$$
This is known as the mean-value property.
\begin{proof}
    We have
    $$\oint_{\partial D_\rho(a)}\frac{f(z)-f(w)}{z-w}\,\mathrm dz=0$$
    by the preceding theorem.
    So we have
    \begin{align*}
        \oint_{\partial D_\rho(a)}\frac{f(z)}{z-w}\,\mathrm dz&=f(w)\oint_{\partial D_\rho(a)}\frac{1}{z-w}\,\mathrm dz\\
        &=f(w)\oint_{\partial D_\rho(a)}\frac{1}{z-a}\,\mathrm dz+f(w)\oint_{\partial D_\rho(a)}\sum_{n=1}^\infty\frac{(w-a)^n}{(z-a)^{n+1}}\,\mathrm dz\\
        &=f(w)2\pi i+f(w)\sum_{n=1}^\infty(w-a)^n\oint_{\partial D_\rho(a)}\frac{1}{(z-a)^{n+1}}\,\mathrm dz\\
        &=f(w)2\pi i
    \end{align*}
    Note that we can change the order of integration and summation since the series (as a geometric series) converges uniformly (which is easy enough to prove).
\end{proof}
So we can prove that a bounded entire function is constant.
\begin{proof}[Proof of Theorem \ref{holo_bdd_const}]
    Let $f$ be a bounded entire function.
    It suffice to assume that $f$ has sublinear growth since it is bounded.
    So $|f(z)|\le C(1+|z|^\alpha)$ for some $C\ge 0$ and $\alpha\in (0,1)$.
    Let $w\in\mathbb C$, by Cauchy integral formula, for any $\rho>|w|$ we have
    $$f(w)=\frac{1}{2\pi i}\oint_{D_\rho(0)}\frac{f(z)}{z-w}\,\mathrm dz$$
    Also
    $$f(0)=\frac{1}{2\pi i}\oint_{D_\rho(0)}\frac{f(z)}{z}\,\mathrm dz$$
    Hence
    \begin{align*}
        |f(w)-f(0)|&=\left|\frac{1}{2\pi i}\oint_{D_\rho(0)}f(z)\left(\frac{1}{z-w}-\frac{1}{z}\right)\,\mathrm dz\right|\\
        &=\frac{|w|}{2\pi}\left|\oint_{D_\rho(0)}\frac{f(z)}{z(z-w)}\,\mathrm dz\right|\\
        &\le|w|\rho\sup_{z\in D_\rho(0)}\frac{|f(z)|}{|z||z-w|}\\
        &=\sup_{z\in D_\rho(0)}\frac{|w||f(z)|}{|z-w|}\\
        &\le\sup_{z\in D_\rho(0)}\frac{|w|C(1+\rho^\alpha)}{|z-w|}\to 0
    \end{align*}
    As $\rho\to\infty$.
    Hence $f(w)=f(0)$.
\end{proof}
\begin{corollary}
    Every nonconstant polynomial with complex coefficient has a root in $\mathbb C$.
\end{corollary}
\begin{proof}
    Assume there is a complex polynomial $p(z)=a_nz^n+\cdots+a_0$ with $n\ge1,a_n\neq 0$ which has no root, then $p$ is never $0$, hence $1/p$ is entire.
    So it suffices to show that $p$ is bounded.
    Note that for $z\neq 0$ we have $|p(z)|=|z^n||a_n+a_{n-1}z^{-1}+\cdots+a_0z^{-n}|\to\infty$ as $|z|\to\infty$.
    Hence we can choose $R>0$ such that $p(z)\ge 1$ for any $|z|>R$, so $p$ is bounded outside the disk $B_R(0)$, but it is also bounded inside it since $B_R(0)$ is compact.
    So by Theorem \ref{holo_bdd_const}, $p$ must be a constant, contradiction.
\end{proof}
\begin{theorem}[Local Maximum Modulus Principle]
    Let $f:D_r(a)\to\mathbb C$ be holomorphic.
    If $|f(z)|\le|f(a)|$ for any $z\in D_r(a)$, then $f$ is constant.
\end{theorem}
\begin{proof}
    We shall use the mean-value property.
    $$f(a)=\int_0^1 f(a+\rho e^{2\pi it})\,\mathrm dt$$
    for any $0<\rho<r$.
    So
    $$|f(a)|=\left|\int_0^1 f(a+\rho e^{2\pi it})\,\mathrm dt\right|\le\sup_{t\in [0,1]}|f(a+\rho e^{2\pi it})|\le|f(a)|$$
    So automatically all inequality must be equality, therefore $|f(a+\rho e^{2\pi it})|$ must be constant.
    But this works for all $\rho\in (0,r)$, so $|f|$ is constant, which implies $f$ being constant by Cauchy-Riemann equation.
\end{proof}
\begin{theorem}[Taylor Series]
    Let $f:D_r(a)\to\mathbb C$ be holomorphic.
    Then $f$ has a convergent power series representation on $D_a(r)$ in the form
    $$f(w)=\sum_{n=0}^\infty c_n(w-a)^n,c_n=\frac{1}{2\pi i}\oint_{\partial D_\rho(a)}\frac{f(z)}{(z-a)^{n+1}}\,\mathrm dz$$
    for $|w|<\rho<r$.
\end{theorem}
Once we have established this, we immediately know that $f$ is infinitely differentiable on $D_a(r)$, therefore $c_n=f^{(n)}(a)/n!$.
\begin{proof}
    For any $w\in D_r(a)$ and any $\rho$ with $|w-a|<\rho<r$ we have
    \begin{align*}
        f(w)&=\frac{1}{2\pi i}\oint_{\partial D_\rho(a)}\frac{f(z)}{z-w}\,\mathrm dz\\
        &=\frac{f(z)}{2\pi i}\oint_{\partial D_\rho(a)}\sum_{n=0}^\infty\frac{(w-a)^n}{(z-a)^{n+1}}\,\mathrm dz\\
        &=\sum_{n=0}^\infty(w-a)^n\frac{1}{2\pi i}\oint_{\partial D_\rho(a)}\frac{f(z)}{(z-a)^{n+1}}\,\mathrm dz
    \end{align*}
    We can change the order of integration since a geometrical series is uniform.
    This gives us the desired series.
\end{proof}
\begin{corollary}
    If $f:U\to\mathbb C$ is holomorphic where $U$ is open, then $f$ has derivative of all orders and they are all holomorphic.
\end{corollary}
\begin{proof}
    Follows directly.
\end{proof}
A function, real or complex, is said to be analytic on an open set if it has a convergent power series representation there.
So for a complex function, being analytic is equivalent to being holomorphic.
But for real functions, even infinite differentiability does not imply a function is analytic, for example
$$f(x)=\begin{cases}
    e^{-1/x^2}\text{, for $x\neq 0$.}\\
    0\text{, for $x=0$.}
\end{cases}$$
From now on, we can use analytic and holomorphic interchangably.
We can also say now that if we decompose $f=u+iv$, then $u,v$ are automatically smooth given $f$ holomorphic.
\begin{theorem}[Morera's Theorem]
    Let a complex function $f:U\to\mathbb C$ be continuous.
    Suppose that its integral is $0$ along any closed curve, then $f$ is holomorphic.
\end{theorem}
\begin{proof}
    The antiderivative exists and is holomorphic, so its derivative $f$ is holomorphic.
\end{proof}
\begin{theorem}[Cauchy's Integral Formula for Derivatives]
    For a holomorphic function $f:D=D_r(a)\to\mathbb C$ and for any $w\in D$ and $|w|<\rho<r$, we have
    $$f^\prime(w)=\frac{1}{2\pi i}\oint_{\partial D_\rho(a)}\frac{f(z)}{(z-w)^2}\,\mathrm dz$$
    More generally,
    $$f^{(n)}(w)=\frac{n!}{2\pi i}\oint_{\partial D_\rho(a)}\frac{f(z)}{(z-w)^{n+1}}\,\mathrm dz$$
\end{theorem}
\begin{proof}
    Consider the function $g(z)=f(z)/(z-w)$ and its derivative gives the $n=1$ case.
    The general case follows from induction.
\end{proof}
\begin{definition}
    Let $U\subset\mathbb C$ be an open set and let $(f_n)$ be a sequence of complex functions.
    We say it converges locally uniformly if for any $a\in U$, there is some $r>0$ such that $(f_n)$ converges uniformly on $D_r(a)\subset U$.
\end{definition}
\begin{proposition}
    $(f_n)$ converges locally uniformly on $U$ iff it converges uniformly on any compact subset of $U$.
\end{proposition}
\begin{proof}
    Easy.
\end{proof}
\begin{theorem}
    Let $f_n:U\to\mathbb C$ be a sequence of holomorphic functions that converges to $f$ locally uniformly on $U$, then $f$ is holomorphic.
    Moreover, $f_n^\prime\to f^\prime$ locally uniformly on $U$.
\end{theorem}
\begin{remark}
    This is totally not true in the real case since we can approximate any continuous function on a closed interval locally uniformly by polynomials (Weierstrass Approximation Theorem).
\end{remark}
\begin{proof}
    $f$ is continuous since continuity is a local property and we already know that the uniform limit of continuous functions is continuous.
    Pick any point $a\in U,r>0$ with $B_r(a)\subset U$ and $f_n\to f$ uniformly on $B_r(a)$, the by Cauchy's Theorem on star-shaped domains, we have
    $$\oint_\gamma f_n(z)\,\mathrm dz=0$$
    for any closed $\gamma$ on the disk.
    Then by uniformity
    $$\oint_\gamma f(z)\,\mathrm dz=\oint_\gamma\lim_{n\to\infty}f_n(z)\,\mathrm dz=\lim_{n\to\infty}\oint_\gamma f_n(z)\,\mathrm dz$$
    So $f$ is holomorphic on the disk.
    The derivative case follows from Cauchy Integral Formula for derivatives.
\end{proof}
\begin{theorem}[Principle of Isolated Zero]
    Let $f:D_R(w)\to\mathbb C$ be holomorphic and not identically zero, then the zeros of $f$ are isolated, that is, there exists $0<r\le R$ such that $f(z)\neq 0$ for any $0<|z-w|<r$.
\end{theorem}
\begin{proof}
    Trivial.
\end{proof}
\begin{remark}
    The zeros of a holomorphic function can have a limit point on the boundary of the domain.
    For example, $f(z)=\sin(1/z)$ on $\mathbb C^\star$.
\end{remark}
\subsection{Analytic Continuation}
By Taylor Series, we know that a holomorphic function $f$ on a disk $D_R(a)$ is completely determined by the values of the derivatives of $f$ at $a$.
Does this generalize to arbitrary domains?
\begin{theorem}
    Let $D'\subset D$ be domains and $f:D'\to\mathbb C$ be analytic.
    Then there is at most one analytic function $g:D\to\mathbb C$ such that $g|_{D'}=f$.
\end{theorem}
Such a function $g$ is called an analytic continuation of $f$ to $D$.
\begin{proof}
    Suppose there are two of the functions $g_1,g_2$ satisfying this condition, then $g_1|_{D'}=g_2|_{D'}=f$ and $h=g_1-g_2$.
    Let $D_0=\{z\in D:\exists r>0, h|_{D_r(z)}\equiv 0\}$.
    Then $D_0$ is open.
    We shall show that it is also closed, which shall finish our proof.
    But $D_1=\{z\in D:\exists n\in\mathbb N, h^{(n)}(z)\neq 0\}=D\setminus D_0$ is also open in $D$, so one of them is empty by connectedness.
    But $D'\subset D_1$ hence $D_1$ is empty, therefore $h\equiv 0\implies g_1=g_2$.
\end{proof}
Actually this is a direct consequence of the Principle of Isolated Zeros, but one want to show that this depends entirely on the existence of Taylor series.
\begin{remark}
    1. The proof depends on the result that the functions have convergent Taylor series at every point, so it is valid for real valued analytic functions.\\
    2. Analytic continuations to larger domains need not always exist.
    For example, the function
    $$f(z)=\sum_{n=0}^\infty z^{n!}$$
    is analytic on $D_1(0)$, but we will prove in example sheet that this cannot be extend to larger domains.
    $\partial D_1(0)$ is called the natural boundary of $f$.
\end{remark}
\begin{corollary}[Identity Principle]
    Let $U\subset \mathbb C$ be a domain and let $f,g:U\to\mathbb C$ be holomorphic and the set $\{z\in U:f(z)=g(z)\}$ has a non-isolated point, then $f\equiv g$.
\end{corollary}
\begin{proof}
    Immediate.
\end{proof}
\begin{corollary}[Global Maximum Principle]
    Let $U\subset\mathbb C$ be a bounded domain and let $f:\bar{U}\to\mathbb C$ be continuous and holomorphic in $U$, then $\sup|f|$ is attained on the boundary $\partial U=\bar U\setminus U^\circ$.
\end{corollary}
\begin{proof}
    Follows from the Local Maximum Principle and Identity Principle.
\end{proof}
\begin{remark}
    Some of the theorems can generalize even further to help us to solve elliptic PDEs as a generlization to Laplace's Equation.
    For example, we also have an analogue of analytic continuation and we can show that harmonic functions in any dimensions are infinitely differentiable.
    Also some analogue of the Local Maximum Modulus Principle also hold.
    So does the mean value property (but this time only for harmonic functions instead of general functions satisfying elliptic PDEs).
    In fact, if the mean value property hold, the function has to be harmonic.
\end{remark}