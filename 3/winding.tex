\section{The Winding Number}
We want to generalize Cauchy's Theorem to non-simply-connected domains and generalize Cauchy Integral Formula to general shapes.
\begin{definition}
    Let $\gamma:[a,b]\to\mathbb C$ be a closed curve and fix $w\in\mathbb C\setminus\operatorname{Im}\gamma$.
    For each $t$, we can write $\gamma(t)=w+r(t)e^{i\theta(t)}$.
    Where $r(t)=|\gamma(t)-w|$.
    If $\gamma$ is piecewise $C^1$, so is $r$.
    If we can find a continuous $\theta(t)$ such that the equation holds for every $t$, then we define the winding number (or index) if $\gamma$ about $w$ by
    $$I(\gamma;w)=\frac{\theta(b)-\theta(a)}{2\pi}$$
\end{definition}
Note that the winding number must be an integer.
It is easy to show that it is well-defined (i.e. independent of the choice of continuous $\theta$), and also hopefully we can show that it always exists.
\begin{lemma}
    If $\gamma$ is a piecewise $C^1$ curve and $w$ not in the image of $\gamma$, then there exists continuous piecewise $C^1$ real function $\theta$ such that
    $$\gamma(t)=w+r(t)e^{i\theta(t)}$$
    where $r(t)=|\gamma(t)-w|$.
\end{lemma}
If $\gamma$ is $C^1$ and the lemma holds, then $\gamma^\prime(t)=r^\prime(t)e^{i\theta(t)}+ir(t)\theta^\prime(r)e^{i\theta(t)}$, which rearranges to give
$$\theta^\prime(t)=\operatorname{Im}\left( \frac{\gamma^\prime(t)}{\gamma(t)-w} \right)\implies\theta(t)=\theta(a)+\operatorname{Im}\left( \int_a^t \frac{\gamma^\prime(x)}{\gamma(x)-w}\,\mathrm dx\right)$$
\begin{proof}
    Let $h(t)=\int_a^t\frac{\gamma^\prime(s)}{\gamma(s)-w}\,\mathrm ds$ where the singularities are skipped when integrate.
    So $h$ is continuous and differentiable wherever $\gamma$ is, where we will have $h^\prime(t)=\gamma^\prime(t)/(\gamma(t)-w)$.
    Except the singularities, we have
    $$\frac{\mathrm d}{\mathrm dt}\left( (\gamma(t)-w)e^{-h(t)} \right)=\gamma^\prime(t)e^{-h(t)}-(\gamma(t)-w)e^{-h(t)}h^\prime(t)=0$$
    So $(\gamma(t)-w)e^{-h(t)}$ is piecewise constant, but it is also continuous, so it is constant.
    $h(a)=0$, so if we let $\alpha$ be the argument of $\gamma(a)-w$, then
    \begin{align*}
        \gamma(t)-w&=(\gamma(a)-w)e^{h(t)}\\
        &=|\gamma(a)-w|e^{i\alpha}e^{\operatorname{Re}h(t)}e^{i\operatorname{Im}h(t)}\\
        &=|\gamma(a)-w|e^{\operatorname{Re}h(t)}e^{i(\alpha+\operatorname{Im}h(t))}
    \end{align*}
    Taking $\theta(t)=\alpha+\operatorname{Im}h(t)$ finishes the proof.
\end{proof}
\begin{corollary}
    If additionally $\gamma(t)$ is a closed curve, then
    $$I(\gamma;w)=\frac{1}{2\pi i}\int_\gamma\frac{\mathrm dz}{z-w}$$
\end{corollary}
\begin{proof}
    Follows directly from the above choice of $\theta$.
\end{proof}
\begin{remark}
    Actually the lemma holds for continuous $\gamma$ (where we want $r,\theta$ to be continuous).
    The proof is exercise.
\end{remark}
\begin{proposition}
    If $\gamma:[a,b]\to D_R(\alpha)$ be a piecewise $C^1$ closed curve and $w\notin D_R(\alpha)$, then $I(\gamma,w)=0$.
\end{proposition}
\begin{proof}
    Write out definition and use Cauchy.
\end{proof}
\begin{proposition}
    The function $w\mapsto I(\gamma,w)$ is locally constant.
\end{proposition}
\begin{proof}
    Immediate from continuity.
\end{proof}
\begin{definition}
    Let $U\subset\mathbb C$ is open.\\
    1. A closed curve $\gamma$ in $U$ is homologous to $0$ if $I(\gamma,w)$ is $0$ for all $w\notin U$.\\
    2. $U$ is simply connected if every closed curve in $U$ is homologous to $0$.
\end{definition}
So a disk is simply connected but an annulus is not.
\begin{theorem}[Cauchy Integral Formula]\label{general_cif}
    Let $U$ be open and let $\gamma:[a,b]\to U$ be closed and homologous to $0$, then for any holomorphic $f:U\to\mathbb C$ we have
    $$\frac{1}{2\pi i}\oint_\gamma\frac{f(z)}{z-w}\,\mathrm dz=I(\gamma,w)f(w)$$
    for any $w\in U$ that is not in the image of $\gamma$.
\end{theorem}
Note that we can assume $U$ is a bounded domain.
In particular, by integrating $F(z)=(z-w)f(z)$, we know from the theorem that
$$\int_\gamma f(z)\,\mathrm dz=0$$
\begin{corollary}
    If $U$ is simply connected, then for any closed curve $\gamma$ in $U$ and any holomorphic function $f$ on $U$, we have
    $$\int_\gamma f(z)\,\mathrm dz=0$$
\end{corollary}
\begin{proof}
    Immediate from the preceding theorem.
\end{proof}
\begin{remark}
    By the theorem, fix $\gamma$, if
    $$\oint_\gamma f(z)\,\mathrm dz=0$$
    holds for functions in the form $f(z)=1/(z-w)$, then it holds for every holomorphic $f$.
\end{remark}
\begin{proposition}
    Suppose $U\subset\mathbb C$ is open and $\phi:U\times [a,b]\to\mathbb C$ is continuous and each $z\mapsto \phi(z,s)$ is holomorphic, then
    $$\int_a^b\phi(z,s)\,\mathrm ds$$
    is holomorphic.
\end{proposition}
\begin{proof}
    Morera's and Fubini's on closed intervals.
\end{proof}
\begin{proof}[Proof of Theorem \ref{general_cif}]
    First define a function $g$ on $U\times U\to\mathbb C$ by
    $$g(z,w)=\begin{cases}
        (f(z)-f(w))/(z-w)\text{, if $z\neq w$}\\
        f^\prime(z)\text{, if $z=w$}
    \end{cases}$$
    which is continuous since $f$ is holomorphic.
    Also define
    $$h(w)=\oint_\gamma g(z,w)\,\mathrm dz,w\in U;h_1(w)=\oint_\gamma\frac{f(z)}{z-w}\,\mathrm dz,w\in\mathbb C\setminus\gamma([a,b])=U_1$$
    Also, $h=h_1$ on $U\cap U_1$ and $U\cup U_1=\mathbb C$ since $\gamma$ is homologous to zero, so we can define
    $$\phi(w)=\begin{cases}
        h(w)\text{, if $w\in U$}\\
        h_1(w)\text{, if $w\in U_1$}
    \end{cases}$$
    Then $\phi$ is entire by the preceding proposition, and since the behaviour of $\phi$ as $|w|\to\infty$ shall be that of $h_1$, and as $I(\gamma;w)=0$ for large enough $|w|$, we have
    $$|\phi(w)|\le\frac{\operatorname{length}(\gamma)\sup_\gamma|f|}{|w|-R}\to 0$$
    as $|w|\to\infty$.
    By Theorem \ref{holo_bdd_const}, $\phi$ is constantly zero, in particular, $h$ is constantly zero on $U$.
    This implies the result.
\end{proof}
The clever part of the proof that we used a global theorem (i.e. Liouville) to show a local result, which is pretty cool.
\begin{definition}
    Let $\gamma_0,\gamma_1:[a,b]\to\mathbb C$ be two closed piecewise $C^1$ curves.
    We say $\gamma_0$ is homotopic to $\gamma_1$ if there is a function $H:[a,b]\times [0,1]\to\mathbb C$ with $H(0,t)=\gamma_0(t),H(1,t)=\gamma_1(t)$, and for each $s\in [0,1]$, $H(t,s)$ is a closed piecewise $C^1$ curve, that is $\forall s\in[0,1],H(s,a)=H(s,b)$.
\end{definition}
\begin{theorem}\label{homotopy}
    For curves $\gamma_0,\gamma_1:[0,1]\to U$ be homotopic and $w\neq U$, we have $I(\gamma_0,w)=I(\gamma_1,w)$.
\end{theorem}
\begin{lemma}
    If $\gamma_0,\gamma_1:[0,1]\to\mathbb C$ are piecewise $C^1$, $w\in\mathbb C$ and
    $$\forall t\in [0,1],|\gamma_0(t)-\gamma_1(t)|<|w-\gamma_1(t)|$$
    then $I(\gamma_0;w)=I(\gamma_1;w)$.
\end{lemma}
\begin{proof}
    Exercise.
\end{proof}
We will show the $C^1$ case, but the general case is also true by similar idea.
\begin{proof}[Proof of Theorem \ref{homotopy}]
    Consider the continuous deformation $H:[0,1]\times [0,1]\to\mathbb C$.
    Now $[0,1]^2$ is compact, so $H$ has compact (hence closed) image and is uniformly continuous.
    So for $w\notin U$, so there is $\epsilon>0$, for any $(s,t)\in [0,1]^2$, $|H(s,t)-w|>2\epsilon$.
    Also there is $n\in\mathbb N$ such that
    $$|s-s'|+|t-t'|\le\frac{1}{n}\implies |H(s,t)-H(s',t')|<\epsilon$$
    Denote $H(s,t)$ by $\gamma_s(t)$, this would mean that for $k=1,2,\ldots,n$, we have $|\gamma_{(k-1)/n}(t)-\gamma_{k/n}(t)|<\epsilon$, but we also have $|w-\gamma_{k/n}(t)|>2\epsilon$, so
    $$|\gamma_{(k-1)/n}(t)-\gamma_{k/n}(t)|<\epsilon<2\epsilon<|w-\gamma_{k/n}(t)|$$
    So by the preceding lemma $I(\gamma_0;w)=I(\gamma_{1/n};w)=\cdots=I(\gamma_1;w)$.
\end{proof}
So a star-shaped domain is simply connected.
Given $\gamma:[0,1]\to U$ and let $p$ be the centre of $U$, then we can shrink $\gamma$ to $p$ in the obvious way.
\begin{remark}
    If a curve is null-homotopic, then by the theorem it is homologous to $0$, but the converse might not be true.
    The usual algebraic topological definition of simple-connectedness is that every closed curve is null-homotopic.
    So $U$ is simply-connected in the algebraic topological way implies that it is simply connected in the complex analysis way in terms of winding number.
    This then implies that Cauchy's Theorem (in simply connected domains) holds.\\
    The reverse implication holds but complex simply-connectedness implying topological simply-connectedness part is not trivial.
\end{remark}